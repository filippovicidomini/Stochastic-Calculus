\argomento{Linear Ordinary Differential Equations}

A \textbf{linear ordinary differential equation (ODE)} of first order has the form:
\[
\frac{dx}{dt} + a(t)x(t) = b(t),
\]
where \( a(t) \) and \( b(t) \) are given functions (typically continuous), and \( x(t) \) is the unknown function.

A \textbf{Cauchy problem} for a linear ODE consists in solving:
\[
\frac{dx}{dt} + a(t)x(t) = b(t), \quad x(t_0) = x_0.
\]

\textbf{Theorem (Existence and Uniqueness):}  
If \( a(t) \) and \( b(t) \) are continuous on an open interval \( I \) containing \( t_0 \), then the Cauchy problem has a unique solution defined on all of \( I \).

\textbf{Proof (Sketch):}  
We rewrite the equation in standard form:
\[
\frac{dx}{dt} = -a(t)x(t) + b(t),
\]
and observe that the right-hand side is continuous and Lipschitz in \( x \) (since it is linear in \( x \)), satisfying the hypotheses of the Picard–Lindelöf theorem. Hence, the solution exists and is unique.

\textbf{General Solution Formula:}  
We can solve the linear ODE using an integrating factor. Define:
\[
\mu(t) := e^{\int_{t_0}^t a(s)\,ds},
\]
which is always positive since the exponential of a continuous function is continuous and never zero.

Multiplying both sides of the ODE by \( \mu(t) \), we get:
\[
\mu(t) \frac{dx}{dt} + \mu(t)a(t)x(t) = \mu(t)b(t).
\]

Using the product rule:
\[
\frac{d}{dt}[\mu(t)x(t)] = \mu(t)b(t).
\]

Integrating both sides from \( t_0 \) to \( t \):
\[
\mu(t)x(t) - \mu(t_0)x_0 = \int_{t_0}^{t} \mu(s)b(s)\,ds.
\]

Solving for \( x(t) \):
\[
x(t) = \frac{1}{\mu(t)}\left( \mu(t_0)x_0 + \int_{t_0}^{t} \mu(s)b(s)\,ds \right).
\]

\textbf{Example:} Solve
\[
\frac{dx}{dt} + 2x = \sin(t), \quad x(0) = 0.
\]

Here, \( a(t) = 2 \), \( b(t) = \sin(t) \), so:
\[
\mu(t) = e^{\int_0^t 2\,ds} = e^{2t}.
\]

Then:
\[
\frac{d}{dt}[e^{2t}x(t)] = e^{2t}\sin(t),
\]
\[
e^{2t}x(t) = \int_0^t e^{2s}\sin(s)\,ds.
\]

Finally:
\[
x(t) = e^{-2t} \int_0^t e^{2s} \sin(s)\,ds.
\]

This expresses the unique solution.