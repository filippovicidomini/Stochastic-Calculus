\section{It\^o--Doeblin Formula}
Let $X_t$ be an It\^o process with dynamics 
\[
  dX_t = \mu_t\,dt + \sigma_t\,dB_t, \quad X_0 = x_0,
\]
where $(B_t)$ is Brownian motion.  For a twice-differentiable function $f(t,x)$ (with $\partial_t f$ continuous and $f$ of class $C^2$ in $x$), the {\em It\^o–Doeblin formula} (also known as It\^o's lemma) gives the stochastic chain rule.  In differential form it reads 
\[
  d\bigl[f(t,X_t)\bigr]
  = \Bigl(f_t(t,X_t) + \mu_t\,f_x(t,X_t) + \tfrac12\sigma_t^2\,f_{xx}(t,X_t)\Bigr)\,dt 
    \;+\;\sigma_t\,f_x(t,X_t)\,dB_t,
\] 
where subscripts denote partial derivatives.  Equivalently, in integral form: 
{\small
\[
  f(t,X_t) - f(0,X_0)
  = \int_0^t\bigl(f_s(s,X_s) + \mu_s f_x(s,X_s) + \tfrac{1}{2} \sigma_s^2 f_{xx}(s,X_s)\bigr)\,ds
  + \int_0^t \sigma_s f_x(s,X_s)\,dB_s.
\]
}
Here the extra term $\tfrac12\sigma_t^2 f_{xx}\,dt$ is the {\em It\^o correction} arising from the quadratic variation of $B_t$.  This formula is the stochastic analogue of the classical chain rule, with the additional $(dB_t)^2 = dt$ term.  It implies that $Y_t=f(t,X_t)$ is itself an It\^o process. In contrast to deterministic calculus (which would give $d f = f_t dt + f_x dX$), It\^o calculus requires $f$ to be $C^{1,2}$ (once in $t$, twice in $x$) and yields the extra second-derivative term.

\paragraph{Example 1: $f(x)=x^2$.}  As a simple illustration, let $Y_t=X_t^2$ with $dX_t=\mu_tdt+\sigma_t dB_t$.  We have $f'(x)=2x$, $f''(x)=2$.  By It\^o's formula,
\[
  d(X_t^2) = 2X_t\,dX_t + d\langle X\rangle_t,
\]
and since $(dX_t)^2 = \sigma_t^2\,dt$, this becomes 
\[
  d(X_t^2) = 2X_t(\mu_tdt + \sigma_t dB_t) + \sigma_t^2 dt 
           = \bigl(2X_t\mu_t + \sigma_t^2\bigr)\,dt + 2X_t\sigma_t\,dB_t.
\]
In other words, the {\em drift} part of $X_t^2$ is $2X_t\mu_t+\sigma_t^2$, not just $2X_t\mu_t$.  Integrating from $0$ to $t$ gives 
\[
  X_t^2 = X_0^2 + \int_0^t(2X_s\mu_s + \sigma_s^2)\,ds + \int_0^t 2X_s\sigma_s\,dB_s.
\]

\paragraph{Example 2: $f(x)=\ln x$ for Geometric Brownian Motion.}  Consider a geometric Brownian motion $X_t$ solving 
\[
  dX_t = \mu\,X_t\,dt + \sigma\,X_t\,dB_t,\quad X_0>0.
\]
Take $Y_t = \ln X_t$.  Here $f'(x)=1/x,\ f''(x)=-1/x^2$.  It\^o's formula gives
\[
  d(\ln X_t) = \frac{1}{X_t}\,dX_t - \frac{1}{2X_t^2}\,(dX_t)^2
  = \frac{1}{X_t}(\mu X_t dt + \sigma X_t dB_t) - \frac12\frac{1}{X_t^2}(\sigma^2 X_t^2 dt).
\]
Simplifying,
\[
  d(\ln X_t) = \mu\,dt + \sigma\,dB_t \;-\; \tfrac12\sigma^2\,dt
            = \Bigl(\mu - \tfrac12\sigma^2\Bigr)dt + \sigma\,dB_t.
\]
Thus 
\[
  \ln X_t = \ln X_0 + \sigma B_t + \Bigl(\mu - \tfrac12\sigma^2\Bigr)t,
\]
and exponentiating yields the familiar solution 
$X_t = X_0 \exp\bigl(\sigma B_t + (\mu-\tfrac12\sigma^2)t\bigr)$. 

\begin{figure}[ht]
\centering
\fbox{\begin{minipage}{0.8\textwidth}
\begin{center}
\textbf{Chain Rule Comparison}
\[
\begin{array}{ll}
\text{Classical:} & df = f_x\,dX \\
\text{It\^o:} & df = f_x\,dX + \tfrac12 f_{xx}\,(dX)^2
\end{array}
\]
\end{center}
\end{minipage}}
\caption{In the stochastic setting, the differential of $f(X_t)$ includes a second-order correction term due to quadratic variation.}
\end{figure}

\section{Comparison of Classical and It\^o Calculus}
\begin{table}[ht]
\centering
\small
\caption{Classico vs. It\^o}
\begin{tabular}{p{3.2cm}p{4.1cm}p{4.1cm}}
\hline
\textbf{Aspetto} & \textbf{Calcolo classico} & \textbf{Calcolo di It\^o} \\
\hline
Regola catena & $df = f_x\,dX$ & $df = f_t dt + f_x dX + \frac12 f_{xx} (dX)^2$ \\
Variazione quadr. & Trascurata & $(dB)^2 = dt$ \\
Integrale & Riemann/Lebesgue & Integrale di It\^o (media quadratica) \\
Reg. prodotto & $d(XY) = X\,dY + Y\,dX$ & $d(XY) = X\,dY + Y\,dX + d\langle X,Y\rangle$ \\
Regolarità $f$ & $C^1$ & $C^{1,2}$ (1 in $t$, 2 in $x$) \\
Contesto & Deterministico & Stocastico (con moto browniano) \\
\hline
\end{tabular}
\end{table}

\section{Why the It\^o Integral is Different}
Brownian motion paths are almost nowhere differentiable and have infinite variation, so standard Riemann (or Lebesgue) integration theory does not apply to $\int B_s\,dB_s$. Instead, the It\^o integral is defined via mean-square limits of adapted Riemann sums. This leads to martingale properties but also introduces extra terms. For example:
\[
  \int_0^t B_s\,dB_s \;=\; \tfrac12B_t^2 - \tfrac12 t,
\]
as can be derived from It\^o's formula applied to $f(x)=x^2$. The $-\tfrac12 t$ term reflects the It\^o correction.