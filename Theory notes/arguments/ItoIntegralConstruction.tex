\argomento{Construction of the Itô Integral}

\textbf{Goal.}  
Define the integral
\[
\int_0^t H_s \, dB_s
\]
for a stochastic process \( (H_s) \) adapted to the filtration \( (\mathcal{F}_s) \) generated by Brownian motion \( (B_s) \), and satisfying appropriate integrability conditions.

\vspace{1em}
\textbf{Step 1: Simple (Elementary) Processes.}  
We begin by defining the Itô integral for simple processes of the form:
\[
H_s = \sum_{i=0}^{n-1} H_i \, \mathbf{1}_{(t_i, t_{i+1}]}(s),
\]
where:
\begin{itemize}
  \item \( 0 = t_0 < t_1 < \cdots < t_n = t \) is a partition of the interval \([0, t]\);
  \item \( H_i \) is \( \mathcal{F}_{t_i} \)-measurable and bounded (i.e., adapted);
  \item \( H_s \) is constant on each subinterval.
\end{itemize}

For such a process, we define:
\[
\int_0^t H_s \, dB_s := \sum_{i=0}^{n-1} H_i (B_{t_{i+1}} - B_{t_i}).
\]

\vspace{1em}
\textbf{Step 2: Extension to Square-Integrable Processes.}  
Let \( H_s \) be an \( \mathcal{F}_s \)-adapted process such that:
\[
\mathbb{E} \left[ \int_0^t H_s^2 \, ds \right] < \infty.
\]
Then there exists a sequence of simple processes \( (H^n_s) \) such that:
\[
\lim_{n \to \infty} \mathbb{E} \left[ \int_0^t |H^n_s - H_s|^2 \, ds \right] = 0.
\]
We define:
\[
\int_0^t H_s \, dB_s := \lim_{n \to \infty} \int_0^t H^n_s \, dB_s,
\]
where the limit is taken in \( L^2(\Omega) \).

\vspace{1em}
\textbf{Properties of the Itô Integral.}
\begin{itemize}
  \item \textbf{Linearity:} The integral is linear in \( H \).
  \item \textbf{Isometry (Itô isometry):}
  \[
  \mathbb{E} \left[ \left( \int_0^t H_s \, dB_s \right)^2 \right] = \mathbb{E} \left[ \int_0^t H_s^2 \, ds \right].
  \]
  \item \textbf{Martingale property:} If \( H \) is adapted and square-integrable, then the process
  \[
  M_t = \int_0^t H_s \, dB_s
  \]
  is a martingale with respect to \( (\mathcal{F}_t) \).
\end{itemize}

\vspace{1em}
\textbf{Interpretation.}  
Unlike Riemann or Riemann–Stieltjes integrals, the Itô integral respects the randomness and lack of smoothness of Brownian motion. It is a \textit{limit of stochastic sums}, where the integrand is evaluated at the left endpoints—this is crucial in distinguishing Itô from Stratonovich calculus.

\textbf{Conclusion.}  
The Itô integral extends classical integration to the stochastic setting. Its construction is based on stepwise processes and extended by density in \( L^2 \), providing a rigorous and practical tool to study stochastic differential equations.