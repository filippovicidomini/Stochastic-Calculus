

\section{Random Walk Exercise}

Given an infinite sequence of coin tosses
\[
\omega = H, T, T, H, H, \ldots
\]
and a scaled random walk \( W^{(n)}(t) \), compute the values:
\[
W^{(16)}\left(\frac{1}{16}\right), \quad W^{(16)}\left(\frac{2}{16}\right), \quad W^{(16)}\left(\frac{1.5}{16}\right)
\]

\section*{Step-by-Step Resolution}

\subsection*{1. Define the random variables \( X_i \)}

For each toss:
\[
X_i =
\begin{cases}
+1 & \text{if the $i$-th toss is Head (H)} \\
-1 & \text{if the $i$-th toss is Tail (T)}
\end{cases}
\]

For the given sequence:
\[
X_1 = +1, \quad X_2 = -1, \quad X_3 = -1, \quad X_4 = +1, \quad X_5 = +1, \ldots
\]

\subsection*{2. Compute the partial sums \( S_k \)}

Define the random walk:
\[
S_k = X_1 + X_2 + \cdots + X_k
\]

For the first few terms:
\[
\begin{aligned}
S_0 &= 0 \\
S_1 &= +1 \\
S_2 &= +1 -1 = 0 \\
S_3 &= +1 -1 -1 = -1 \\
S_4 &= -1 +1 = 0 \\
S_5 &= 0 +1 = +1 \\
\end{aligned}
\]

\subsection*{3. Use the formula for the scaled random walk (with interpolation)}

The scaled random walk is defined as:
\[
W^{(n)}(t) = \frac{1}{\sqrt{n}} \cdot S_{\lfloor nt \rfloor}
\quad \text{(if $t$ is a multiple of $1/n$)}
\]

If \( t \) is not a multiple of \( 1/n \), we use **linear interpolation**:
\[
W^{(n)}(t) = \frac{1}{\sqrt{n}} \left[ (1 - (nt - \lfloor nt \rfloor)) \cdot S_{\lfloor nt \rfloor} + (nt - \lfloor nt \rfloor) \cdot S_{\lfloor nt \rfloor + 1} \right]
\]

\paragraph{Examples:}
\begin{itemize}
\item \( \displaystyle W^{(16)}\left(\frac{1}{16}\right) = \frac{1}{\sqrt{16}} \cdot S_1 = \frac{1}{4} \cdot 1 = 0.25 \)
\item \( \displaystyle W^{(16)}\left(\frac{2}{16}\right) = \frac{1}{\sqrt{16}} \cdot S_2 = \frac{1}{4} \cdot 0 = 0 \)
\item \( \displaystyle W^{(16)}\left(\frac{1.5}{16}\right) = \frac{1}{\sqrt{16}} \left[ 0.5 \cdot S_1 + 0.5 \cdot S_2 \right] = \frac{1}{4} \cdot (0.5 \cdot 1 + 0.5 \cdot 0) = 0.125 \)
\end{itemize}

\section*{Conclusion}

To solve this type of exercise:

\begin{enumerate}
    \item Translate each toss into \( X_i = \pm 1 \)
    \item Compute the partial sums \( S_k \)
    \item Use the formula \( W^{(n)}(t) = \frac{1}{\sqrt{n}} \cdot S_{\lfloor nt \rfloor} \)
\end{enumerate}

This procedure approximates the trajectory of Brownian motion using a discrete random walk derived from coin tosses.