\argomento{Itô Isometry and Application to the Vasicek Model}

\textbf{Itô Isometry.}  
Let \( H = (H_t)_{t \geq 0} \) be an adapted process such that:
\[
\mathbb{E} \left[ \int_0^T H_s^2 \, ds \right] < \infty.
\]
Then the Itô integral \( I(t) := \int_0^t H_s \, dB_s \) satisfies:
\[
\mathbb{E} \left[ \left( \int_0^t H_s \, dB_s \right)^2 \right] = \mathbb{E} \left[ \int_0^t H_s^2 \, ds \right], \quad \forall t \in [0, T].
\]
This is a fundamental property of the Itô integral. It means that the mapping \( H \mapsto \int H \, dB \) is an isometry from \( L^2([0,T] \times \Omega) \) into \( L^2(\Omega) \).

\vspace{1em}
\subsection{Application: Variance in the Vasicek Interest Rate Model}

The Vasicek model for the short rate \( r_t \) is given by the stochastic differential equation:
\[
dr_t = a(b - r_t) \, dt + \sigma \, dB_t, \quad r_0 \in \mathbb{R}.
\]
This is a linear SDE, and its solution is:
\[
r_t = r_0 e^{-at} + b(1 - e^{-at}) + \sigma \int_0^t e^{-a(t - s)} \, dB_s.
\]

Let us compute the variance of \( r_t \). Since the deterministic part has zero variance, it remains to compute:
\[
\operatorname{Var}(r_t) = \operatorname{Var} \left( \sigma \int_0^t e^{-a(t - s)} \, dB_s \right)
= \sigma^2 \mathbb{E} \left[ \left( \int_0^t e^{-a(t - s)} \, dB_s \right)^2 \right].
\]

Applying Itô's isometry:
\[
\operatorname{Var}(r_t) = \sigma^2 \int_0^t e^{-2a(t - s)} \, ds
= \sigma^2 \int_0^t e^{-2au} \, du
= \frac{\sigma^2}{2a} \left( 1 - e^{-2at} \right).
\]

\textbf{Interpretation:}  
As \( t \to \infty \), the variance of \( r_t \) tends to \( \frac{\sigma^2}{2a} \), which shows the long-term stability of the Vasicek model around the mean-reverting level \( b \).

\textbf{Conclusion:}  
The Itô isometry allows one to compute second moments of stochastic integrals and is especially useful in finance, for instance, in understanding the volatility and risk of interest rate models.