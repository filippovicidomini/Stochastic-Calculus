\argomento{The Logistic Equation}

The \textbf{logistic equation} is a classical model for population growth that includes the effect of limited resources. It is given by:
\[
\frac{dx}{dt} = rx\left(1 - \frac{x}{K}\right),
\]
where:
- \( x(t) \) is the population at time \( t \),
- \( r > 0 \) is the intrinsic growth rate,
- \( K > 0 \) is the carrying capacity of the environment.

\vspace{1em}
\noindent
\textbf{Interpretation:}
- For small \( x \), the term \( \left(1 - \frac{x}{K}\right) \approx 1 \), so the equation behaves like exponential growth: \( \frac{dx}{dt} \approx rx \).
- As \( x \to K \), \( \frac{dx}{dt} \to 0 \): growth slows down and the population stabilizes at the carrying capacity.
- If \( x > K \), the growth rate becomes negative: the population decreases back toward \( K \).

The logistic model captures the idea that growth is self-limiting due to competition for resources.

\vspace{1em}
\noindent
\textbf{Qualitative analysis:}
The equilibria are found by solving \( \frac{dx}{dt} = 0 \), which yields:
\[
x = 0 \quad \text{and} \quad x = K.
\]
- \( x = 0 \) is an unstable equilibrium (any small positive population grows away from zero).
- \( x = K \) is a stable equilibrium (the population stabilizes at \( K \)).

We can also analyze the sign of the derivative:
- If \( 0 < x < K \), then \( \frac{dx}{dt} > 0 \): population increases.
- If \( x > K \), then \( \frac{dx}{dt} < 0 \): population decreases.

\vspace{1em}
\noindent
\textbf{Explicit solution:}

To solve the logistic equation:
\[
\frac{dx}{dt} = rx\left(1 - \frac{x}{K}\right),
\]
we separate variables:
\[
\frac{dx}{x(1 - x/K)} = r\,dt.
\]

We use partial fractions:
\[
\frac{1}{x(1 - x/K)} = \frac{1}{x} + \frac{1}{K - x} \cdot \frac{1}{K}.
\]

Actually, it's better to write:
\[
\frac{1}{x(1 - x/K)} = \frac{1}{K} \left( \frac{1}{x} + \frac{1}{K - x} \right).
\]

Then integrate both sides:
\[
\int \left( \frac{1}{x} + \frac{1}{K - x} \right) dx = \int r\,dt.
\]

\[
\ln|x| - \ln|K - x| = rK t + C.
\]

This simplifies to:
\[
\ln\left(\frac{x}{K - x}\right) = rK t + C.
\]

Exponentiating both sides:
\[
\frac{x}{K - x} = A e^{rK t}, \quad \text{where } A = e^C.
\]

Solving for \( x(t) \):
\[
x(t) = \frac{K A e^{rK t}}{1 + A e^{rK t}} = \frac{K}{1 + B e^{-rK t}}, \quad \text{where } B = \frac{1}{A}.
\]

Determine \( B \) using the initial condition \( x(0) = x_0 \):
\[
x_0 = \frac{K}{1 + B}, \quad \Rightarrow B = \frac{K - x_0}{x_0}.
\]

\textbf{Final explicit solution:}
\[
x(t) = \frac{K}{1 + \left( \frac{K - x_0}{x_0} \right) e^{-rK t}}.
\]

\vspace{1em}
\noindent
\textbf{Conclusion:}
- The population grows toward the carrying capacity \( K \).
- The growth is initially exponential, then slows down due to competition.
- The logistic equation models self-limited growth more realistically than exponential models.