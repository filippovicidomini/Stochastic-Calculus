\argomento{Lebesgue Integrability}

A function \( f: \mathbb{R} \to \mathbb{R} \) (or more generally, \( f: X \to \mathbb{R} \) where \( (X, \mathcal{A}, \mu) \) is a measure space) is said to be \textbf{Lebesgue integrable} over a measurable set \( A \subseteq \mathbb{R} \) if

\[
\int_A |f(x)| \, d\mu(x) < \infty.
\]

That is, the absolute value of \( f \) must have a finite Lebesgue integral.

\section*{Intuition}

Lebesgue integrability ensures that the ``area under the curve'' of \( f \) (in a generalized sense) is finite. Unlike Riemann integration, which partitions the domain, Lebesgue integration partitions the codomain (i.e., the range of function values).

This allows the Lebesgue integral to handle:
\begin{itemize}
    \item Functions with infinitely many discontinuities,
    \item Functions that are not bounded,
    \item Functions defined on sets of complicated structure.
\end{itemize}

\subsection*{Example}

Define \( f: [0,1] \to \mathbb{R} \) as follows:

\[
f(x) =
\begin{cases}
1 & \text{if } x \in \mathbb{Q} \cap [0,1], \\
0 & \text{if } x \in \mathbb{R} \setminus \mathbb{Q} \cap [0,1].
\end{cases}
\]

This function is:
\begin{itemize}
    \item \textbf{Not Riemann integrable}, because it is discontinuous at every point of \([0,1]\).
    \item \textbf{Lebesgue integrable}, because the set of rational numbers \( \mathbb{Q} \cap [0,1] \) has Lebesgue measure zero. Therefore,
    \[
    \int_0^1 f(x) \, dx = 0.
    \]
\end{itemize}

\subsection*{Comparison: Riemann vs Lebesgue}

\begin{center}
\begin{tabular}{|l|c|c|}
\hline
\textbf{Property} & \textbf{Riemann Integrable} & \textbf{Lebesgue Integrable} \\
\hline
Requires continuity & Almost everywhere & No \\
\hline
Handles discontinuities & Poorly & Very well \\
\hline
Based on partitions of & Domain (x-axis) & Codomain (y-axis) \\
\hline
Uses measure theory & No & Yes \\
\hline
Suitable for advanced analysis & No & Yes \\
\hline
\end{tabular}
\end{center}

\subsection*{Conclusion}

Lebesgue integrability is a more general and robust notion of integration, fundamental in modern analysis, probability theory, and partial differential equations. It allows integration of a wider class of functions and aligns naturally with the structure of measure spaces.
