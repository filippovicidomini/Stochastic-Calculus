\argomento{Basic Definitions in Probability Theory}

\textbf{a) \(\sigma\)-algebra}

A \textbf{\(\sigma\)-algebra} \( \mathcal{F} \) on a set \( \Omega \) is a collection of subsets of \( \Omega \) such that:
\begin{itemize}
  \item \( \Omega \in \mathcal{F} \),
  \item If \( A \in \mathcal{F} \), then \( A^c \in \mathcal{F} \),
  \item If \( A_1, A_2, \dots \in \mathcal{F} \), then \( \bigcup_{n=1}^\infty A_n \in \mathcal{F} \).
\end{itemize}

\textit{Interpretation:} A \(\sigma\)-algebra represents a collection of events for which we can define probabilities in a consistent way, and it is closed under the basic operations of probability theory (complementation, countable unions).

\vspace{1em}
\textbf{b) \(\sigma\)-algebra generated by a random variable}

Given a random variable \( X : \Omega \to \mathbb{R} \), the \textbf{\(\sigma\)-algebra generated by \( X \)} is:
\[
\sigma(X) = \{ X^{-1}(B) \subseteq \Omega \mid B \in \mathcal{B}(\mathbb{R}) \},
\]
where \( \mathcal{B}(\mathbb{R}) \) is the Borel \(\sigma\)-algebra on \( \mathbb{R} \). 
In mathematics, a Borel set is any subset of a topological space that can be formed from its open sets (or, equivalently, from closed sets) through the operations of countable union, countable intersection, and relative complement. Borel sets are named after Émile Borel.




\textit{Interpretation:} \( \sigma(X) \) is the collection of all events that can be determined by observing the value of \( X \). It represents the “information” contained in the variable \( X \).

\vspace{1em}
\textbf{c) Filtration}

A \textbf{filtration} is a family \( (\mathcal{F}_t)_{t \geq 0} \) of \(\sigma\)-algebras such that:
\[
\mathcal{F}_s \subseteq \mathcal{F}_t \subseteq \mathcal{F}, \quad \text{for all } 0 \leq s \leq t.
\]

\textit{Interpretation:} A filtration represents the evolution of information over time. \( \mathcal{F}_t \) contains all the events that can be observed up to time \( t \).

\vspace{1em}
\textbf{d) Stochastic process}

A \textbf{stochastic process} is a family of random variables \( (X_t)_{t \in T} \) defined on a probability space \( (\Omega, \mathcal{F}, \mathbb{P}) \), where each \( X_t : \Omega \to \mathbb{R} \) (or more generally to some measurable space) for \( t \in T \).

\textit{Interpretation:} A stochastic process describes the evolution of a random quantity over time. Each \( X_t \) gives the state of the system at time \( t \).

\vspace{1em}
\textbf{e) Adapted stochastic process}

Let \( (\mathcal{F}_t)_{t \in T} \) be a filtration. A stochastic process \( (X_t)_{t \in T} \) is said to be \textbf{adapted} to the filtration if, for each \( t \in T \), the random variable \( X_t \) is \( \mathcal{F}_t \)-measurable.

\textit{Interpretation:} An adapted process is one where the value of \( X_t \) at time \( t \) only depends on the information available up to time \( t \), and not on future events.