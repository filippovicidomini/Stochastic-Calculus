\argomento{Analyzing Ordinary Differential Equations (ODEs)}

This chapter introduces basic techniques for analyzing first-order ordinary differential equations (ODEs), focusing on existence and uniqueness theorems, and the effect of varying initial conditions.

\subsection*{Local and Global Existence and Uniqueness}
Consider the Cauchy problem:
\begin{equation} \label{eq:IVP}
\begin{cases}
\displaystyle \frac{dx}{dt} = f(t, x), \\
x(t_0) = x_0,
\end{cases}
\end{equation}
where $f: \mathbb{R} \times \mathbb{R} \to \mathbb{R}$ is a continuous function.

\subsubsection*{Local Existence and Uniqueness Theorem (Picard-Lindel\"of)}
If $f(t, x)$ is continuous and Lipschitz continuous in $x$ in a neighborhood of $(t_0, x_0)$, then there exists a unique local solution $x(t)$ defined on some interval containing $t_0$.

\textbf{Lipschitz condition in $x$:}
\[|f(t, x_1) - f(t, x_2)| \leq L |x_1 - x_2| \text{ for all } x_1, x_2 \text{ in a neighborhood.}\]

\subsubsection*{Global Existence}
If the solution remains bounded and $f(t,x)$ remains continuous and satisfies a global Lipschitz condition in $x$, the solution can be extended to all $t$ in $\mathbb{R}$. Otherwise, the solution may blow up in finite time.

\textbf{Criterion:} If $|f(t,x)| \leq \alpha(|x|)$ for a continuous, non-decreasing function $\alpha$ such that:
\[\int^\infty \frac{1}{\alpha(s)}\, ds = \infty,\]
then blow-up is avoided, and global existence is ensured.

\subsection*{Monotonicity with Respect to Initial Data}
Let $x_1(t)$ and $x_2(t)$ be solutions to the same ODE \eqref{eq:IVP} with initial conditions $x_1(t_0) = x_1^0 < x_2^0 = x_2(t_0)$. Then:

\subsubsection*{Order-Preserving Property}
If $f(t, x)$ is non-decreasing in $x$ (i.e., $\frac{\partial f}{\partial x} \geq 0$), then:
\[x_1(t) < x_2(t) \quad \text{for all } t \text{ where the solutions exist.}\]

This property allows us to analyze how the solution responds to variations in the initial value. This is particularly useful for ODEs arising in population dynamics, chemical reactions, or economics.

\subsection{Examples}

\subsubsection*{Example 1: Logistic Equation}
\begin{equation*}
\frac{dx}{dt} = rx(1 - x/K), \quad x(0) = x_0,
\end{equation*}
where $r, K > 0$.

\begin{itemize}
  \item $f(t,x) = rx(1 - x/K)$ is locally Lipschitz in $x$ $\Rightarrow$ local existence.
  \item The RHS is a polynomial in $x$ $\Rightarrow$ global existence.
  \item $\frac{\partial f}{\partial x} = r(1 - 2x/K)$, so the monotonicity depends on the value of $x$.
  \item For $x_0 < K$, $x(t)$ increases and asymptotically approaches $K$.
\end{itemize}

\subsubsection*{Example 2: Linear ODE}
\begin{equation*}
\frac{dx}{dt} = ax + b, \quad x(0) = x_0,
\end{equation*}
with constants $a, b \in \mathbb{R}$.

\begin{itemize}
  \item Solution: $x(t) = (x_0 + \frac{b}{a})e^{at} - \frac{b}{a}$ if $a \neq 0$, or $x(t) = x_0 + bt$ if $a = 0$.
  \item $f(t,x)$ is linear and globally Lipschitz $\Rightarrow$ global existence and uniqueness.
  \item $\frac{\partial f}{\partial x} = a$, so monotonicity depends on the sign of $a$.
\end{itemize}

\subsubsection*{Example 3: Nonlinear Explosive ODE}
\begin{equation*}
\frac{dx}{dt} = x^2, \quad x(0) = x_0 > 0.
\end{equation*}

\begin{itemize}
  \item Solution: $x(t) = \frac{x_0}{1 - x_0 t}$.
  \item Solution blows up in finite time $t = 1/x_0$ $\Rightarrow$ no global existence.
  \item $f(t,x)$ is increasing in $x$, so order-preserving with respect to initial conditions.
\end{itemize}

\subsubsection*{Example 4: Time-dependent Coefficient}
\begin{equation*}
\frac{dx}{dt} = t x, \quad x(0) = x_0.
\end{equation*}

\begin{itemize}
  \item Solution: $x(t) = x_0 e^{t^2/2}$.
  \item $f(t,x) = t x$ is locally Lipschitz in $x$, smooth in $t$ $\Rightarrow$ global existence.
  \item $\frac{\partial f}{\partial x} = t$, so monotonicity in initial condition holds when $t \geq 0$.
\end{itemize}

\subsubsection*{Example 5: Nonlinear ODE with Exponential and Logarithmic Terms}
Consider the ODE:
\begin{equation*}
\frac{dx}{dt} = x(\ln^2(x)) e^{\ln(x)}, \quad t > 0,
\end{equation*}
coupled with the initial condition $x(0) = x_0 > 0$.

Let us define:
\[ f(x) = x (\ln^2(x)) e^{\ln(x)} = x (\ln^2(x)) x = x^2 \ln^2(x). \]

\subsubsection*{(a) Stationary Solutions and Monotonicity}
Stationary solutions satisfy $f(x) = 0$. Since $x^2 > 0$ for $x > 0$, we need:
\[ \ln(x) = 0 \Rightarrow x = 1. \]
So, the only stationary solution is $x \equiv 1$.

Now, analyze the sign of $f(x) = x^2 \ln^2(x)$ for different values of $x_0$:
\begin{itemize}
  \item If $x_0 = 1$, then $\ln(x) = 0 \Rightarrow \dot{x} = 0$ (stationary).
  \item If $0 < x_0 < 1$, then $\ln(x) < 0$, so $\ln^2(x) > 0 \Rightarrow \dot{x} > 0$.
  \item If $x_0 > 1$, then $\ln(x) > 0 \Rightarrow \dot{x} > 0$.
\end{itemize}
Thus, for all $x_0 \neq 1$, the solution is strictly increasing.

\subsubsection*{(b) Existence and Uniqueness for $x_0 = 1/3$}
Define $f(x) = x^2 \ln^2(x)$. This function is continuous for $x > 0$ and differentiable. Let's compute the derivative:
\[ f'(x) = 2x \ln^2(x) + 2x \ln(x), \]
which is continuous for $x > 0$ $\Rightarrow$ $f$ is locally Lipschitz on $(0, \infty)$.

Hence, by the Picard–Lindel"of theorem, there exists a unique local solution for any $x_0 > 0$, including $x_0 = 1/3$.

Global existence: since $f(x) = x^2 \ln^2(x)$ grows polynomially for $x \to \infty$ and goes to $0$ as $x \to 0^+$, there is no blow-up in finite time, and the solution exists globally.

\subsubsection*{(c) Compute the Solution for $x_0 = 1/3$}
We separate variables:
\begin{align*}
\frac{dx}{x^2 \ln^2(x)} &= dt.
\end{align*}
Let $u = \ln(x) \Rightarrow du = \frac{1}{x} dx \Rightarrow dx = x du = e^u du$.
Then:
\begin{align*}
\frac{dx}{x^2 \ln^2(x)} &= \frac{e^u du}{e^{2u} u^2} = \frac{du}{e^u u^2}.
\end{align*}
Thus,
\[ \int \frac{du}{e^u u^2} = t + C. \]
This integral does not have an elementary closed-form, but defines $u(t) = \ln(x(t))$ implicitly. Numerical or special function techniques are needed to write $x(t)$ explicitly.

Therefore, the solution can be expressed implicitly via:
\[ \int_{\ln(x_0)}^{\ln(x(t))} \frac{du}{e^u u^2} = t. \]
