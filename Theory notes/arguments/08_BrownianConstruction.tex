\argomento{Construction of Brownian Motion}

\textbf{Definition.}  
A stochastic process \( (B_t)_{t \geq 0} \) is called a \textbf{standard Brownian motion} (or \textbf{Wiener process}) if it satisfies the following properties:
\begin{enumerate}
  \item \( B_0 = 0 \) almost surely;
  \item \( (B_t) \) has \textbf{independent increments}: for any \( 0 \leq t_0 < t_1 < \dots < t_n \), the random variables \( B_{t_1} - B_{t_0}, B_{t_2} - B_{t_1}, \dots, B_{t_n} - B_{t_{n-1}} \) are independent;
  \item The increments are \textbf{stationary and Gaussian}: for all \( 0 \leq s < t \),  
  \[
  B_t - B_s \sim \mathcal{N}(0, t - s);
  \]
  \item The sample paths \( t \mapsto B_t(\omega) \) are almost surely continuous.
\end{enumerate}

\textit{Interpretation:}  
Brownian motion models the limit of a symmetric random walk in continuous time. It is a fundamental object in probability theory and plays a central role in stochastic calculus, statistical physics, and financial mathematics.

\vspace{1em}
\textbf{Construction via Scaled Random Walks (Donsker's Invariance Principle):}  

Let \( (X_i)_{i \geq 1} \) be a sequence of i.i.d.\ real-valued random variables with zero mean and unit variance:
\[
\mathbb{E}[X_i] = 0, \qquad \mathrm{Var}(X_i) = 1.
\]
Define the partial sums:
\[
S_k = \sum_{i=1}^k X_i.
\]
For each \( n \in \mathbb{N} \), define the rescaled and interpolated process:
\[
W_n(t) := \frac{1}{\sqrt{n}} S_{\lfloor nt \rfloor} + \frac{1}{\sqrt{n}} (nt - \lfloor nt \rfloor) X_{\lfloor nt \rfloor + 1}, \quad t \in [0,1].
\]

Then, by the \textbf{Donsker invariance principle} (functional central limit theorem), the sequence of processes \( (W_n(t))_{t \in [0,1]} \) converges in distribution in \( C[0,1] \) (with the uniform topology) to standard Brownian motion \( (B_t)_{t \in [0,1]} \) as \( n \to \infty \).

\vspace{1em}
\textbf{Alternative Construction via Karhunen–Loève Expansion:}

Brownian motion on the interval \( [0,1] \) can also be constructed as an infinite series:
\[
B_t = \sum_{n=1}^\infty Z_n \sqrt{\lambda_n} e_n(t),
\]
where:
\begin{itemize}
  \item \( (Z_n) \) is a sequence of i.i.d.\ standard normal random variables;
  \item \( (\lambda_n, e_n(t)) \) are the eigenvalues and eigenfunctions of the covariance operator of Brownian motion, typically:
  \[
  \lambda_n = \frac{1}{((n - 1/2)\pi)^2}, \quad e_n(t) = \sqrt{2} \sin\left( (n - 1/2)\pi t \right),
  \]
  for Dirichlet boundary conditions.
\end{itemize}
This representation highlights the Gaussian structure and regularity properties of Brownian paths.

\vspace{1em}
\textbf{Remarks:}
\begin{itemize}
  \item Brownian motion is a continuous-time martingale with respect to its natural filtration.
  \item Sample paths of \( B_t \) are almost surely continuous everywhere but nowhere differentiable.
  \item Brownian motion is the unique (in law) continuous process with independent and stationary Gaussian increments.
  \item It satisfies the scaling property: for any \( c > 0 \), the process \( (B_{ct})_{t \geq 0} \) has the same distribution as \( (\sqrt{c} B_t)_{t \geq 0} \).
\end{itemize}