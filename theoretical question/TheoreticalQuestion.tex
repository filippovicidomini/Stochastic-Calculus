\documentclass[12pt,a4paper]{book}

% Pacchetti utili
\usepackage[utf8]{inputenc}  % Per la codifica in UTF-8
\usepackage[T1]{fontenc}     % Font encoding
\usepackage{lmodern}         % Font Latin Modern
\usepackage{amsmath, amssymb, amsthm, mathtools, bm}
\usepackage{hyperref}        % Collegamenti ipertestuali
\usepackage{enumitem}        % Miglior gestione degli elenchi
\usepackage{geometry}        % Controllo margini
%fancy
\usepackage{fancyhdr}        % Intestazioni e piè di pagina
\geometry{a4paper, margin=2.5cm}

% Intestazioni e piè di pagina
\pagestyle{fancy}
\fancyhf{}
\fancyhead[L]{Theoretical Questions Stochastic Calculus}
\fancyhead[R]{\thepage}
%\fancyfoot[C]{\textit{Confidential - Do Not Distribute}}

% Ambienti per teoremi e definizioni
\newtheorem{definition}{Definition}[chapter]
\newtheorem{theorem}{Theorem}[chapter]
\newtheorem{example}{Example}[chapter]
\newtheorem{lemma}{Lemma}[chapter]
\newtheorem{proposition}{Proposition}[chapter]
\theoremstyle{remark}
\newtheorem{remark}{Remark}[chapter]

% Comandi personalizzati per probabilità e processi stocastici
\newcommand{\PP}{\mathbb{P}}          % probabilità
\newcommand{\EE}{\mathbb{E}}          % valore atteso
\newcommand{\QQ}{\mathbb{Q}}          % misura risk-neutral
\newcommand{\RR}{\mathbb{R}}
\newcommand{\NN}{\mathbb{N}}
\newcommand{\F}{\mathcal{F}}          % sigma-algebra
\newcommand{\Filtr}[1]{\{\mathcal{F}_{#1}\}} % filtrazione
\newcommand{\Var}{\mathrm{Var}}       % varianza
\newcommand{\Cov}{\mathrm{Cov}}       % covarianza
\newcommand{\indic}{\mathds{1}}       % funzione indicatrice

% Altri comandi utili
\DeclarePairedDelimiter{\abs}{\lvert}{\rvert}
\DeclarePairedDelimiter{\norm}{\lVert}{\rVert}
\DeclarePairedDelimiter{\ang}{\langle}{\rangle}
\newcommand{\law}{\stackrel{d}{=}}    % uguaglianza in distribuzione
\newcommand{\given}{\,\middle|\,} % condizionamento

% Differenziali e calcolo stocastico
\newcommand{\dd}{\mathrm{d}}
\newcommand{\dt}{\,\mathrm{d}t}
\newcommand{\dx}{\,\mathrm{d}x}
\newcommand{\dW}{\,\mathrm{d}W_t}
\newcommand{\Ito}{It\^o}
\newcommand{\BM}{Brownian motion}

\begin{document}

\begin{titlepage}
    \centering
    % Logo dell'università (se vuoi inserirlo, basta caricare l'immagine)
    % \includegraphics[width=0.25\textwidth]{logo.png}\par\vspace{1cm}
    
    {\scshape Università Ca' Foscari Venezia \par}
 
    \vspace{2cm}
    \vfill
    {\Huge\bfseries Theoretical Questions on Stochastic Calculus \par}
    \vspace{1.5cm}
    {\Large\itshape Exam Preparation Booklet\par}
    
    \vspace{2cm}
    {\Large Course: Stochastic Calculus for Finance \par}
    \vspace{0.5cm}
 
    
    \vfill
    
    {\large Student: Filippo Vicidomini \par}
    {\large Master’s Degree in Engineering Physics\par}
    
    \vfill
    
    % Data
    {\large Academic Year 2024--2025 \par}
    
\end{titlepage}



\section{Question 7}
\textbf{Give the definition of the martingale property for a stochastic process and interpret it. Give suitable examples of stochastic processes with this property.}

\subsection*{Answer}
Let $(\Omega, \F, \PP)$ be a probability space and $\Filtr{t}$ a filtration. A stochastic process $X(t)$ adapted to $\Filtr{t}$ is called a \textbf{martingale} if:

\begin{enumerate}[label=\roman*)]
    \item $\EE[\abs{X(t)}] < \infty$ for all $t$;
    \item For all $s < t$, 
    \[
        \EE[X(t) \mid \F_s] = X(s).
    \]
\end{enumerate}

\subsection*{Interpretation}
A martingale represents a \textbf{fair game}: given the information available up to time $s$, the best prediction of the value at time $t$ is exactly the current value $X(s)$. This means the process has no drift: it does not systematically increase or decrease.

Consequently, $\EE[X(t)] = \EE[X(0)]$ for all $t$.

\subsection*{Examples}
\begin{itemize}
    \item \textbf{Symmetric random walk}: $M_n = \sum_{j=1}^n X_j$ with $X_j = \pm 1$ with equal probability, is a martingale with respect to the natural filtration.
    \item \textbf{Brownian motion} $W(t)$: is a martingale with respect to its natural filtration.
    \item \textbf{It\^o integrals}: if $\Delta(t)$ is adapted and square-integrable, then 
    \[
    I(t) = \int_0^t \Delta(s)\,\dd W(s)
    \]
    is a martingale with zero mean.
\end{itemize}

\subsection*{Counterexample}
A Geometric Brownian Motion 
\[
S(t) = S(0) e^{(\alpha - \tfrac{1}{2}\sigma^2)t + \sigma W(t)}
\]
is not a martingale if $\alpha \neq 0$, since it has exponential drift. However, under the risk-neutral measure $\QQ$, the discounted price $e^{-rt}S(t)$ is a martingale. This property is fundamental in financial mathematics (e.g., Black--Scholes model).


\newpage
\section{Question 8}
\textbf{Describe the construction of a Brownian motion.}

\subsection*{Answer}
A Brownian motion, also known as a Wiener process, is a stochastic process $W(t)$ defined on a probability space $(\Omega, \F, \PP)$ that satisfies the following properties:

\begin{enumerate}[label=\roman*)]
    \item $W(0) = 0$ almost surely;
    \item $W(t)$ has independent increments: for $0 \leq t_0 < t_1 < \cdots < t_n$, the increments 
    \[
        W(t_1) - W(t_0), \; W(t_2) - W(t_1), \ldots, W(t_n) - W(t_{n-1})
    \]
    are independent random variables;
    \item $W(t)$ has Gaussian increments: for $s < t$, the increment $W(t) - W(s)$ is normally distributed with mean $0$ and variance $t-s$;
    \item $W(t)$ has continuous trajectories almost surely.
\end{enumerate}

\subsection*{Construction via Random Walks}
One can construct a Brownian motion as the limit of suitably rescaled symmetric random walks:

\begin{itemize}
    \item Consider a sequence $(X_j)_{j\geq 1}$ of i.i.d. random variables with
    \[
        \PP(X_j = 1) = \PP(X_j = -1) = \tfrac{1}{2}.
    \]
    \item Define the partial sums (a symmetric random walk):
    \[
        M_k = \sum_{j=1}^k X_j, \quad M_0 = 0.
    \]
    Then $\EE[M_k] = 0$, $\Var(M_k) = k$.
    \item Define the scaled random walk:
    \[
        W^{(n)}(t) = \frac{1}{\sqrt{n}} M_{\lfloor nt \rfloor}, \quad t \geq 0.
    \]
    \item As $n \to \infty$, the processes $W^{(n)}(t)$ converge in distribution to a process $W(t)$ that satisfies the above four properties.
\end{itemize}

The limit process $W(t)$ is called a \textbf{Brownian motion}.

\subsection*{Properties}
From this construction, Brownian motion inherits:
\begin{itemize}
    \item Mean zero: $\EE[W(t)] = 0$;
    \item Variance linear in time: $\Var(W(t)) = t$;
    \item Independent, Gaussian increments;
    \item Quadratic variation: $[W,W]_t = t$;
    \item Martingale property: $\EE[W(t) \mid \F_s] = W(s)$ for $s < t$.
\end{itemize}

\subsection*{Interpretation}
Brownian motion models continuous-time randomness:
\begin{itemize}
    \item In physics, it describes the irregular motion of particles suspended in a fluid.
    \item In finance, it underlies models of asset price fluctuations (e.g., geometric Brownian motion in the Black--Scholes framework).
\end{itemize}





\end{document}