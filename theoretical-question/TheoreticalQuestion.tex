\documentclass[12pt,a4paper]{book}

% Pacchetti utili
\usepackage[utf8]{inputenc}  % Per la codifica in UTF-8
\usepackage[T1]{fontenc}     % Font encoding
\usepackage{lmodern}         % Font Latin Modern
\usepackage{amsmath, amssymb, amsthm, mathtools, bm}
%\usepackage{hyperref}        % Collegamenti ipertestuali ma con rettangolo rosso
\usepackage{enumitem}        % Miglior gestione degli elenchi
\usepackage{geometry}        % Controllo margini
%fancy
\usepackage{fancyhdr}        % Intestazioni e piè di pagina
\usepackage[hidelinks]{hyperref}% togli il contorno rosso nell'indice
\usepackage{tocloft} % per controllo indice
\geometry{a4paper, margin=2.5cm}

% Intestazioni e piè di pagina
\pagestyle{fancy}
\fancyhf{}
\fancyhead[L]{Theoretical Questions Stochastic Calculus}
\fancyhead[R]{\thepage}
%\fancyfoot[C]{\textit{Confidential - Do Not Distribute}}
% add a link on the botton right to the page with the table of contents
\fancyfoot[R]{\hyperref[sec:Contents]{Table of Contents}}

% Ambienti per teoremi e definizioni
\newtheorem{definition}{Definition}[chapter]
\newtheorem{theorem}{Theorem}[chapter]
\newtheorem{example}{Example}[chapter]
\newtheorem{lemma}{Lemma}[chapter]
\newtheorem{proposition}{Proposition}[chapter]
\theoremstyle{remark}
\newtheorem{remark}{Remark}[chapter]

% Comandi personalizzati per probabilità e processi stocastici
\newcommand{\PP}{\mathbb{P}}          % probabilità
\newcommand{\EE}{\mathbb{E}}          % valore atteso
\newcommand{\QQ}{\mathbb{Q}}          % misura risk-neutral
\newcommand{\RR}{\mathbb{R}}
\newcommand{\NN}{\mathbb{N}}
\newcommand{\F}{\mathcal{F}}          % sigma-algebra
\newcommand{\Filtr}[1]{\{\mathcal{F}_{#1}\}} % filtrazione
\newcommand{\Var}{\mathrm{Var}}       % varianza
\newcommand{\Cov}{\mathrm{Cov}}       % covarianza
\newcommand{\indic}{\mathds{1}}       % funzione indicatrice
% comando Normal for normal distribution
\newcommand{\Normal}{\mathcal{N}}

% Altri comandi utili
\DeclarePairedDelimiter{\abs}{\lvert}{\rvert} % 
\DeclarePairedDelimiter{\norm}{\lVert}{\rVert} % 
\DeclarePairedDelimiter{\ang}{\langle}{\rangle} %
\newcommand{\law}{\stackrel{d}{=}}    % uguaglianza in distribuzione
\newcommand{\given}{\,\middle|\,} % condizionamento

% Differenziali e calcolo stocastico
\newcommand{\dd}{\mathrm{d}}
\newcommand{\dt}{\,\mathrm{d}t}
\newcommand{\dx}{\,\mathrm{d}x}
\newcommand{\dW}{\,\mathrm{d}W_t}
\newcommand{\Ito}{It\^o}
\newcommand{\BM}{Brownian motion}

\begin{document}

\begin{titlepage}
    \centering
    % Logo dell'università (se vuoi inserirlo, basta caricare l'immagine)
    % \includegraphics[width=0.25\textwidth]{logo.png}\par\vspace{1cm}
    
    {\scshape Università Ca' Foscari Venezia \par}
 
    \vspace{2cm}
    \vfill
    {\Huge\bfseries Theoretical Questions on Stochastic Calculus \par}
    \vspace{1.5cm}
    {\Large\itshape Exam Preparation Booklet\par}
    
    \vspace{2cm}
    {\Large Course: Stochastic Calculus for Finance \par}
    \vspace{0.5cm}
 
    
    \vfill
    
    {\large Student: Filippo Vicidomini \par}
    {\large Master’s Degree in Engineering Physics\par}
    
    \vfill
    
    % Data
    {\large Academic Year 2024--2025 \par}
    
\end{titlepage}

% Pagine iniziali senza numeri o con numeri romani
\pagenumbering{roman}

\tableofcontents

\clearpage
\label{sec:Contents}
% Riparti la numerazione da 1, in arabi
\pagenumbering{arabic}
\setcounter{page}{1}

\newpage
\section{Question 1 -- ODEs and Their Solutions}
\textbf{Explain what an ODE is, and what it means for a function $x(t)$ to be a (classical) solution. 
State the theorems that you know and that establish: (a) existence of a solution; (b) local existence and uniqueness of a solution; (c) global existence and uniqueness of a solution. 
Compare them and explain, give suitable examples.}

\subsection*{Answer}

\subsubsection*{Definition of ODE}
An \textbf{ordinary differential equation (ODE)} is an equation where the unknown is a function $x(t)$ of one real variable (often time $t$), and the function appears together with its derivatives.  
A general first–order ODE can be written in normal form as
\[
x'(t) = f(t,x(t)), \quad t \geq t_0.
\]

\subsubsection*{Classical Solution}
A function $x:[t_0,T]\to\RR$ is a \textbf{classical solution} to the Cauchy problem
\[
\begin{cases}
x'(t) = f(t,x(t)), \quad t \geq t_0, \\
x(t_0) = x_0,
\end{cases}
\]
if:
\begin{enumerate}[label=\roman*)]
    \item $x$ is differentiable on $[t_0,T]$;
    \item it satisfies $x'(t) = f(t,x(t))$ for all $t \in [t_0,T]$;
    \item it satisfies the initial condition $x(t_0) = x_0$.
\end{enumerate}

\subsubsection*{Theorems of Existence and Uniqueness}
\begin{itemize}
    \item \textbf{Peano’s Existence Theorem:} If $f(t,x)$ is continuous in a neighborhood of $(t_0,x_0)$, then there exists at least one solution $x(t)$ to the Cauchy problem on some interval around $t_0$. (Existence, but not uniqueness).
    
    \item \textbf{Picard–Lindelöf (Cauchy–Lipschitz) Theorem:} If $f(t,x)$ is continuous in $t$ and \emph{Lipschitz} continuous in $x$, then there exists a \emph{unique} local solution to the Cauchy problem. 
    
    \item \textbf{Global Existence and Uniqueness:} If the assumptions above hold on the entire domain (e.g. $f$ is globally Lipschitz or satisfies suitable growth conditions preventing blow–up), then the solution can be extended uniquely to all $t \geq t_0$.
\end{itemize}

\subsubsection*{Comparison}
\begin{itemize}
    \item Peano theorem ensures existence but possibly many solutions.
    \item Picard–Lindelöf ensures both existence and uniqueness, but only locally in time.
    \item Global results require additional conditions to extend the solution to all times.
\end{itemize}

\subsubsection*{Examples}
\begin{itemize}
    \item \textbf{Non-uniqueness (Peano):} 
    \[
    x'(t) = \sqrt{x(t)}, \quad x(0)=0.
    \] 
    Both $x(t)\equiv 0$ and $x(t) = \tfrac{t^2}{4}$ are solutions, showing non-uniqueness since $f(x)=\sqrt{x}$ is not Lipschitz at $0$.

    \item \textbf{Unique local (and global) solution (Picard–Lindelöf):} 
    \[
    x'(t) = t+x, \quad x(0)=1.
    \]
    Here $f(t,x)=t+x$ is Lipschitz in $x$. The unique solution is
    \[
    x(t) = Ce^t - t - 1.
    \]

    \item \textbf{Global existence:} 
    \[
    x'(t) = -x(t), \quad x(0)=x_0.
    \]
    Solution: $x(t)=x_0e^{-t}$, which exists uniquely for all $t\geq 0$.
\end{itemize}













\newpage
\section{Question 2 -- Linear ODEs}
\textbf{Say what a linear ODE is. Show that, under suitable assumptions on the coefficients (exemplify), a Cauchy problem for a linear ODE has a unique solution. Prove that such solution is given by the known solution formula.}

\subsection*{Answer}

\subsubsection*{Definition}
A \textbf{first–order linear ODE} is an equation of the form
\[
x'(t) = a(t)x(t) + b(t), \quad t \ge t_0,
\]
where $a,b:I\to\RR$ are given real functions on an interval $I\ni t_0$. The associated \textbf{Cauchy problem} is
\[
\begin{cases}
x'(t) = a(t)x(t) + b(t),\\
x(t_0)=x_0.
\end{cases}
\]

\subsubsection*{Existence and Uniqueness (Picard–Lindel\"of)}
If $a(\cdot)$ and $b(\cdot)$ are continuous on $I$, then $f(t,x)=a(t)x+b(t)$ is continuous and Lipschitz in $x$ on compatti: 
\[
|f(t,x)-f(t,y)|=|a(t)||x-y| \le L|x-y| \quad (t\in K\Subset I).
\]
Hence the Cauchy problem admits a \textbf{unique local} solution; if $a,b$ are continuous on all of $I$ and no blow-up occurs, the solution extends \textbf{uniquely} on $I$.

\subsubsection*{Derivation (Proof via integrating factor)}
Consider
\[
x'(t)-a(t)x(t)=b(t).
\]
Let the integrating factor be
\[
\mu(t)=\exp\!\Big(\int_{t_0}^t a(s)\,\dd s\Big),\qquad \mu(t)>0,\;\mu'(t)=a(t)\mu(t).
\]
Multiply the ODE by $\mu(t)$:
\[
\mu(t)x'(t)-\mu(t)a(t)x(t)=\mu(t)b(t).
\]
By the product rule,
\[
\frac{\dd}{\dd t}\big(\mu(t)x(t)\big)=\mu(t)b(t).
\]
Integrate from $t_0$ to $t$:
\[
\mu(t)x(t)-\mu(t_0)x(t_0)=\int_{t_0}^t \mu(r)b(r)\,\dd r.
\]
Since $\mu(t_0)=1$, $x(t_0)=x_0$, we obtain the \textbf{solution formula}
\[
\boxed{\;
x(t)=\exp\!\Big(\int_{t_0}^t a(s)\,\dd s\Big)\left[
x_0+\int_{t_0}^t \exp\!\Big(-\int_{t_0}^r a(u)\,\dd u\Big)\,b(r)\,\dd r
\right].\;}
\]

\subsubsection*{Verification (Plug-in)}
Set $\mu(t)=\exp\!\big(\int_{t_0}^t a\big)$ and write
\[
x(t)=\mu(t)\Big(x_0+\int_{t_0}^t \mu(r)^{-1}b(r)\,\dd r\Big).
\]
Differentiate:
\[
x'(t)=\mu'(t)\Big(x_0+\int_{t_0}^t \mu(r)^{-1}b(r)\,\dd r\Big)+\mu(t)\cdot \mu(t)^{-1}b(t).
\]
Since $\mu'(t)=a(t)\mu(t)$, we get
\[
x'(t)=a(t)\mu(t)\Big(\cdots\Big)+b(t)=a(t)x(t)+b(t),
\]
quindi $x$ soddisfa l’ODE. Inoltre $x(t_0)=\mu(t_0)\big(x_0+0\big)=x_0$. Per unicità (Picard–Lindel\"of), questa è \emph{la} soluzione del problema di Cauchy.

\subsubsection*{Example}
Let
\[
x'(t)=2x(t)+t,\qquad x(0)=1.
\]
Here $a(t)=2$, $b(t)=t$, $\mu(t)=e^{2t}$. Thus
\[
x(t)=e^{2t}\left(1+\int_0^t e^{-2r}\,r\,\dd r\right)
= e^{2t}\left(1-\tfrac12 t e^{-2t}-\tfrac14 e^{-2t}+\tfrac14\right).
\]
This (unique) solution is defined for all $t\in\RR$.













\newpage
\section{Question 6 -- Measure Theory}
\textbf{State and interpret the definition of:  
a) sigma algebra;  
b) sigma algebra generated by a random variable;  
c) filtration;  
d) stochastic process;  
e) stochastic process adapted to a filtration.}

\subsection*{Answer}

\subsubsection*{a) Sigma algebra}
Let $\Omega$ be a sample space. A \textbf{$\sigma$-algebra} $\F$ on $\Omega$ is a collection of subsets of $\Omega$ such that:
\begin{enumerate}[label=\roman*)]
    \item $\Omega \in \F$;
    \item If $A \in \F$, then $A^c \in \F$ (closed under complementation);
    \item If $\{A_n\}_{n=1}^\infty \subseteq \F$, then $\bigcup_{n=1}^\infty A_n \in \F$ (closed under countable unions).
\end{enumerate}
By De Morgan’s laws, $\F$ is also closed under countable intersections.  

\textbf{Interpretation:} a $\sigma$-algebra represents the collection of events that can be “observed” or “measured” in a probabilistic experiment.  

\textbf{Example:} On $\RR$, the Borel $\sigma$-algebra $\mathcal{B}(\RR)$ is generated by all open intervals $(a,b)$.

---

\subsubsection*{b) Sigma algebra generated by a random variable}
Given a random variable $X:\Omega \to \RR$, the \textbf{$\sigma$-algebra generated by $X$}, denoted $\sigma(X)$, is the smallest $\sigma$-algebra such that $X$ is measurable. Formally,
\[
\sigma(X) = \{ X^{-1}(B) : B \in \mathcal{B}(\RR)\}.
\]

\textbf{Interpretation:} $\sigma(X)$ contains exactly the events that can be described in terms of the knowledge of $X$.

\textbf{Example:} If $X(\omega)=1$ when a coin toss is Head and $0$ otherwise, then $\sigma(X)=\{\emptyset, \Omega, \{X=1\}, \{X=0\}\}$.

---

\subsubsection*{c) Filtration}
A \textbf{filtration} $\{\F_t\}_{t\ge0}$ is an increasing family of $\sigma$-algebras:
\[
\F_s \subseteq \F_t \quad \text{for all } 0 \leq s \leq t.
\]

\textbf{Interpretation:} $\F_t$ represents the information available up to time $t$. As time progresses, information increases.  

\textbf{Example:} For a Brownian motion $W(t)$, the \emph{natural filtration} is $\F_t=\sigma(W(s):0\leq s \leq t)$, i.e. all events determined by the past trajectory of $W$ up to time $t$.

---

\subsubsection*{d) Stochastic process}
A \textbf{stochastic process} is a family $\{X(t)\}_{t\ge0}$ of random variables defined on a common probability space $(\Omega, \F, \PP)$.  

\textbf{Interpretation:} $X(t)$ describes the random evolution of a system in time.  
- Fixing $t$, $X(t)$ is a random variable on $\Omega$.  
- Fixing $\omega \in \Omega$, $t \mapsto X(t,\omega)$ is a trajectory (sample path).

\textbf{Example:} A random walk $M_n=\sum_{j=1}^n X_j$ with i.i.d. $\pm1$ steps is a stochastic process indexed by $n\in\NN$.

---

\subsubsection*{e) Adapted stochastic process}
A stochastic process $\{X(t)\}_{t\ge0}$ is \textbf{adapted} to a filtration $\{\F_t\}$ if, for each $t$, $X(t)$ is $\F_t$-measurable.

\textbf{Interpretation:} At time $t$, the value $X(t)$ depends only on the information available up to $t$, not on the future.  

\textbf{Example:} The Brownian motion $W(t)$ is adapted to its natural filtration $\{\F_t\}$, since $W(t)$ is $\F_t$-measurable by construction.













\newpage
\section{Question 7 -- Martingales}
\textbf{Give the definition of the martingale property for a stochastic process and interpret it. Give suitable examples of stochastic processes with this property.}

\subsection*{Answer}
Let $(\Omega, \F, \PP)$ be a probability space and $\Filtr{t}$ a filtration. A stochastic process $X(t)$ adapted to $\Filtr{t}$ is called a \textbf{martingale} if:

\begin{enumerate}[label=\roman*)]
    \item $\EE[\abs{X(t)}] < \infty$ for all $t$;
    \item For all $s < t$, 
    \[
        \EE[X(t) \mid \F_s] = X(s).
    \]
\end{enumerate}

\subsection*{Interpretation}
A martingale represents a \textbf{fair game}: given the information available up to time $s$, the best prediction of the value at time $t$ is exactly the current value $X(s)$. This means the process has no drift: it does not systematically increase or decrease.

Consequently, $\EE[X(t)] = \EE[X(0)]$ for all $t$.

\subsection*{Examples}
\begin{itemize}
    \item \textbf{Symmetric random walk}: $M_n = \sum_{j=1}^n X_j$ with $X_j = \pm 1$ with equal probability, is a martingale with respect to the natural filtration.
    \item \textbf{Brownian motion} $W(t)$: is a martingale with respect to its natural filtration.
    \item \textbf{It\^o integrals}: if $\Delta(t)$ is adapted and square-integrable, then 
    \[
    I(t) = \int_0^t \Delta(s)\,\dd W(s)
    \]
    is a martingale with zero mean.
\end{itemize}

\subsection*{Counterexample}
A Geometric Brownian Motion 
\[
S(t) = S(0) e^{(\alpha - \tfrac{1}{2}\sigma^2)t + \sigma W(t)}
\]
is not a martingale if $\alpha \neq 0$, since it has exponential drift. However, under the risk-neutral measure $\QQ$, the discounted price $e^{-rt}S(t)$ is a martingale. This property is fundamental in financial mathematics (e.g., Black--Scholes model).


\newpage
\section{Definizione di Processo Stocastico}

Un \textbf{processo stocastico} è una famiglia di variabili aleatorie
\[
\{X(t) \; : \; t \in T\}
\]
definite su uno stesso spazio di probabilità $(\Omega, \mathcal{F}, \mathbb{P})$, 
dove $T \subseteq \mathbb{R}_+$ è l’insieme dei tempi. 
Per ogni $t \in T$, $X(t)$ è una variabile aleatoria a valori in $\mathbb{R}$ (o in $\mathbb{R}^d$). 

\begin{itemize}
  \item \textbf{Prospettiva per tempo fissato}: fissato $t$, $X(t)$ descrive lo stato del sistema al tempo $t$. 
  \item \textbf{Prospettiva per esito fissato}: fissato $\omega \in \Omega$, la mappa
  \[
  t \mapsto X(t,\omega)
  \]
  è detta \emph{cammino campione} (sample path o traiettoria), ed è una funzione deterministica di $t$. 
\end{itemize}

Per descrivere l’evoluzione temporale si introduce una \textbf{filtrazione} 
$\{\mathcal{F}_t\}_{t \geq 0}$, cioè una famiglia crescente di $\sigma$-algebre
($\mathcal{F}_s \subseteq \mathcal{F}_t$ se $s \le t$). 
Un processo $X(t)$ si dice \emph{adattato} se, per ogni $t$, la variabile $X(t)$ 
è $\mathcal{F}_t$-misurabile, ossia determinabile sulla base delle informazioni disponibili fino a $t$.











\newpage
\section{Question 8 -- Brownian Motion}
\textbf{Describe the construction of a Brownian motion.}

\subsection*{Answer}
A Brownian motion, also known as a Wiener process, is a stochastic process $W(t)$ defined on a probability space $(\Omega, \F, \PP)$ that satisfies the following properties:

\begin{enumerate}[label=\roman*)]
    \item $W(0) = 0$ almost surely;
    \item $W(t)$ has independent increments: for $0 \leq t_0 < t_1 < \cdots < t_n$, the increments 
    \[
        W(t_1) - W(t_0), \; W(t_2) - W(t_1), \ldots, W(t_n) - W(t_{n-1})
    \]
    are independent random variables;
    \item $W(t)$ has Gaussian increments: for $s < t$, the increment $W(t) - W(s)$ is normally distributed with mean $0$ and variance $t-s$;
    \item $W(t)$ has continuous trajectories almost surely.
\end{enumerate}

\subsection*{Construction via Random Walks}
One can construct a Brownian motion as the limit of suitably rescaled symmetric random walks:

\begin{itemize}
    \item Consider a sequence $(X_j)_{j\geq 1}$ of i.i.d. random variables with
    \[
        \PP(X_j = 1) = \PP(X_j = -1) = \tfrac{1}{2}.
    \]
    \item Define the partial sums (a symmetric random walk):
    \[
        M_k = \sum_{j=1}^k X_j, \quad M_0 = 0.
    \]
    Then $\EE[M_k] = 0$, $\Var(M_k) = k$.
    \item Define the scaled random walk:
    \[
        W^{(n)}(t) = \frac{1}{\sqrt{n}} M_{\lfloor nt \rfloor}, \quad t \geq 0.
    \]
    \item As $n \to \infty$, the processes $W^{(n)}(t)$ converge in distribution to a process $W(t)$ that satisfies the above four properties.
\end{itemize}

The limit process $W(t)$ is called a \textbf{Brownian motion}.

\subsection*{Properties}
From this construction, Brownian motion inherits:
\begin{itemize}
    \item Mean zero: $\EE[W(t)] = 0$;
    \item Variance linear in time: $\Var(W(t)) = t$;
    \item Independent, Gaussian increments;
    \item Quadratic variation: $[W,W]_t = t$;
    \item Martingale property: $\EE[W(t) \mid \F_s] = W(s)$ for $s < t$.
\end{itemize}

\subsection*{Interpretation}
Brownian motion models continuous-time randomness:
\begin{itemize}
    \item In physics, it describes the irregular motion of particles suspended in a fluid.
    \item In finance, it underlies models of asset price fluctuations (e.g., geometric Brownian motion in the Black--Scholes framework).
\end{itemize}










\newpage
\section{Question 9 -- Random Walks}
\textbf{Describe the construction of a random walk and of a scaled random walk. Show that a Brownian motion can be obtained as a limit of scaled random walks.}

\subsection*{Answer}

\subsubsection*{Random Walk}
Let $\{X_j\}_{j\geq 1}$ be a sequence of i.i.d. random variables with
\[
\PP(X_j = 1) = \PP(X_j = -1) = \tfrac{1}{2}.
\]
Define the partial sums
\[
M_k = \sum_{j=1}^k X_j, \quad M_0=0.
\]
The process $\{M_k\}_{k\in\NN}$ is called a \textbf{symmetric random walk}.  
Properties:
\begin{itemize}
    \item $\EE[M_k] = 0$, $\Var(M_k) = k$.
    \item Increments are independent and stationary: $M_{n+m}-M_n \sim \Normal(0,m)$.
    \item $M_k$ is a martingale with respect to the natural filtration.
\end{itemize}

\subsubsection*{Scaled Random Walk}
To approach a continuous-time process, rescale both time and space:
\[
W^{(n)}(t) = \frac{1}{\sqrt{n}} M_{\lfloor nt \rfloor}, \quad t \ge 0.
\]
Interpretation:
\begin{itemize}
    \item Time is accelerated by factor $n$ (steps of size $1/n$).
    \item Space is scaled down by $1/\sqrt{n}$ (variance normalisation).
\end{itemize}
Thus $W^{(n)}(t)$ is a piecewise constant, right–continuous process with jumps $\pm 1/\sqrt{n}$ at times $k/n$.

\subsubsection*{Limit Process: Brownian Motion}
By Donsker’s invariance principle (or functional Central Limit Theorem),
\[
W^{(n)}(t) \;\; \xrightarrow{d}\;\; W(t), \quad \text{as } n\to\infty,
\]
where $W(t)$ is a \textbf{Brownian motion}.  

\textbf{Proof idea:}
\begin{itemize}
    \item Finite-dimensional distributions: by the Central Limit Theorem, for fixed $t$, 
    \[
    W^{(n)}(t) = \frac{1}{\sqrt{n}}M_{\lfloor nt \rfloor} \;\law\; \Normal(0,t).
    \]
    \item Independence of increments: inherited from independence of $X_j$.
    \item Continuous trajectories: obtained in the limit (the $W^{(n)}$ are piecewise constant, but converge in distribution to a continuous process).
\end{itemize}

\subsubsection*{Conclusion}
A Brownian motion $W(t)$ is obtained as the scaling limit of a symmetric random walk.  
\[
W(t) = \lim_{n\to\infty} W^{(n)}(t) \quad \text{in distribution}.
\]

\subsection*{Example}
Simulating many paths of a scaled random walk with large $n$, the trajectories approximate continuous Brownian paths with variance $t$ and independent Gaussian increments.








\newpage
\section{Question 10 -- Properties of Brownian Motion}
\textbf{List the properties of a Brownian Motion. Comment on consequences of (at least two of) such properties, giving examples of applications.}

\subsection*{Answer}

\subsubsection*{Definition}
A \textbf{Brownian motion} (or Wiener process) $\{W(t)\}_{t\ge 0}$ on a probability space $(\Omega,\F,\PP)$ with filtration $\{\F_t\}$ is a stochastic process such that:

\begin{enumerate}[label=\roman*)]
    \item $W(0)=0$ almost surely;
    \item Independent increments: for $0\le t_0 < t_1 < \cdots < t_n$, the increments $W(t_j)-W(t_{j-1})$ are independent;
    \item Stationary Gaussian increments: for $s<t$, 
    \[
    W(t)-W(s) \sim \Normal(0,t-s);
    \]
    \item Almost surely continuous trajectories $t\mapsto W(t,\omega)$;
    \item Quadratic variation: $[W,W]_t = t$;
    \item $\{W(t)\}$ is a martingale with respect to $\{\F_t\}$.
\end{enumerate}

\subsubsection*{Consequences and Applications}

\paragraph{1. Independent and Gaussian increments.}
This property implies that the process has the \emph{Markov property}: the future evolution depends only on the present, not the past.  
\emph{Application:} In finance, this justifies the modeling of asset prices as functions of Brownian motion (e.g. geometric Brownian motion). The independence of increments makes simulation and option pricing tractable.

\paragraph{2. Continuity and nowhere differentiability.}
Brownian paths are continuous but almost surely nowhere differentiable. This prevents interpreting $dW/dt$ in the classical sense, motivating the development of Itô calculus.  
\emph{Application:} In physics, this models erratic particle trajectories (Einstein’s model of molecular diffusion). In finance, it justifies stochastic differentials like
\[
dS(t) = \mu S(t)\,dt + \sigma S(t)\,dW(t).
\]

\paragraph{3. Quadratic variation $[W]_t = t$.}
This fundamental property distinguishes Brownian motion from deterministic differentiable functions. It leads to the extra $\tfrac{1}{2}f''$ term in Itô’s formula.  
\emph{Application:} Pricing via Black–Scholes model relies on Itô’s formula, where the quadratic variation produces the diffusion term in the PDE.

\paragraph{4. Martingale property.}
Since $\EE[W(t)\mid\F_s]=W(s)$, the Brownian motion is a martingale.  
\emph{Application:} In risk–neutral valuation, discounted asset prices must be martingales under the risk–neutral measure. Brownian motion is the core driving noise ensuring absence of arbitrage.

\subsubsection*{Summary}
Brownian motion is the canonical continuous–time stochastic process, whose properties (independent Gaussian increments, continuity, quadratic variation, martingale property) make it the fundamental building block for stochastic calculus, diffusion models, and modern financial mathematics.




\newpage
\section{Question 12 -- Diffusion Processes}
\textbf{Give the definition of a diffusion process. Explain the necessity of an Itô Calculus for the study of evolution of a stochastic process.}

\subsection*{Answer}

\subsubsection*{Definition of a Diffusion Process}
A \textbf{diffusion process} $\{X(t)\}_{t\ge 0}$ is a continuous-time stochastic process defined as the solution of a stochastic differential equation (SDE) of the form
\[
dX(t) = \mu(t,X(t))\,dt + \sigma(t,X(t))\,dW(t), \quad X(0)=x_0,
\]
where:
\begin{itemize}
    \item $W(t)$ is a Brownian motion;
    \item $\mu(t,x)$ is the \emph{drift coefficient}, governing the deterministic trend;
    \item $\sigma(t,x)$ is the \emph{diffusion coefficient}, scaling the random noise.
\end{itemize}
Thus, a diffusion is a continuous Markov process whose local dynamics are described by a drift and a diffusion term.

\subsubsection*{Necessity of Itô Calculus}
Ordinary calculus is not sufficient to study stochastic processes like diffusions because:
\begin{enumerate}[label=\roman*)]
    \item Paths of Brownian motion are almost surely continuous but nowhere differentiable, so $W'(t)$ does not exist in the classical sense;
    \item Quadratic variation of Brownian motion is nonzero: $[W]_t = t$. This breaks the rules of standard calculus (where higher-order terms vanish).
\end{enumerate}

\textbf{Itô Calculus} provides:
\begin{itemize}
    \item A rigorous definition of the stochastic integral
    \[
    \int_0^t \sigma(s,X(s))\,dW(s),
    \]
    for adapted, square-integrable processes $\sigma$;
    \item The \textbf{Itô formula}, a stochastic analogue of the chain rule, which includes an extra term due to quadratic variation:
    \[
    df(X(t)) = f_x(X(t))\,dX(t) + \tfrac{1}{2} f_{xx}(X(t))\,\sigma^2(t,X(t))\,dt.
    \]
\end{itemize}

\subsubsection*{Applications}
\begin{itemize}
    \item In \textbf{physics}, diffusions describe random particle motion (Einstein’s model of molecular diffusion).
    \item In \textbf{finance}, asset prices are modeled as diffusions (e.g. geometric Brownian motion), and Itô calculus underpins the derivation of option pricing models like Black–Scholes.
\end{itemize}

\subsubsection*{Conclusion}
Diffusion processes generalize deterministic dynamical systems by adding stochastic noise. Their study requires Itô calculus, since classical tools of analysis are not valid for processes driven by Brownian motion.

\newpage
\section{Question 13 -- Itô Integral}
\textbf{Describe the construction of the Itô integral for a stochastic process.}

\subsection*{Answer}

\subsubsection*{Aim}
We want to give a rigorous meaning to the stochastic integral
\[
I(t) = \int_0^t \Delta(s)\,dW(s),
\]
where $W(t)$ is a Brownian motion and $\Delta(s)$ is a stochastic process adapted to the filtration $\{\F_s\}$.

---

\subsubsection*{Step 1: Constant integrands}
For a constant process $\Delta(s) \equiv C$, define
\[
\int_0^T C\,dW(s) := C\,(W(T)-W(0)).
\]

---

\subsubsection*{Step 2: Simple processes}
For a simple (piecewise constant, adapted) process
\[
\Delta(s) = \sum_{j=0}^{n-1} c_j \,\chi_{[t_j,t_{j+1})}(s), \quad 0=t_0<\dots<t_n=T,
\]
define
\[
\int_0^T \Delta(s)\,dW(s) := \sum_{j=0}^{n-1} c_j \big(W(t_{j+1})-W(t_j)\big).
\]

---

\subsubsection*{Step 3: General processes}
If $\Delta(s)$ is progressively measurable and square-integrable, i.e.
\[
\EE\!\left[\int_0^T \Delta(s)^2 \,ds\right] < \infty,
\]
then we approximate $\Delta(s)$ by a sequence of simple processes $\Delta^n(s)$ in $L^2([0,T]\times\Omega)$.  
The Itô integral is defined as the $L^2$–limit:
\[
\int_0^T \Delta(s)\,dW(s) := \lim_{n\to\infty} \int_0^T \Delta^n(s)\,dW(s).
\]

---

\subsubsection*{Properties}
The Itô integral $I(t)=\int_0^t \Delta(s)\,dW(s)$ satisfies:
\begin{itemize}
    \item \textbf{Linearity:} $\int (a\Delta_1+b\Delta_2)\,dW = a\int\Delta_1\,dW + b\int\Delta_2\,dW$;
    \item \textbf{Isometry (Itô isometry):}
    \[
    \EE\!\left[\left(\int_0^T \Delta(s)\,dW(s)\right)^2\right] = \EE\!\left[\int_0^T \Delta^2(s)\,ds\right];
    \]
    \item $\EE\!\big[\int_0^T \Delta(s)\,dW(s)\big] = 0$ (zero mean);
    \item $I(t)$ is a martingale with respect to $\{\F_t\}$.
\end{itemize}

---

\subsubsection*{Interpretation}
The Itô integral extends the notion of integration to stochastic processes.  
- The approximation by simple processes reflects the idea of integrating “stepwise predictable strategies” against Brownian motion.  
- It is fundamental in defining stochastic differential equations and in deriving Itô’s formula.  
- In finance, it represents the gains from trading strategies where $\Delta(t)$ is the number of risky assets held at time $t$.







\newpage
\section{Question 14 -- Itô Integral Properties}
\textbf{List the properties of the It\^o integral. Give an interpretation, or comment, at least of some properties. Show one significant application.}

\subsection*{Answer}

Let $I(t)=\int_0^t \Delta(s)\,\dd W(s)$ with $\Delta$ progressively measurable and $\EE\!\big[\int_0^T \Delta^2(s)\,\dd s\big]<\infty$.

\subsubsection*{Main properties}
\begin{enumerate}[label=\roman*)]
\item \textbf{Linearity:} $\displaystyle \int_0^t (a\Delta_1+b\Delta_2)\,\dd W = a\int_0^t \Delta_1\,\dd W + b\int_0^t \Delta_2\,\dd W.$
\item \textbf{Isometria di It\^o:} $\displaystyle \EE\!\left[\left(\int_0^t \Delta\,\dd W\right)^2\right]=\EE\!\left[\int_0^t \Delta^2\,\dd s\right].$
\item \textbf{Zero mean:} $\displaystyle \EE\!\left[\int_0^t \Delta\,\dd W\right]=0.$
\item \textbf{Martingala:} $I(t)$ è una martingala w.r.t. $\{\F_t\}$ e ha traiettorie continue.
\item \textbf{Covarianza/Prodotto scalare:} per $\Delta,\Gamma\in L^2_{\text{prog}}$,
\[
\EE\!\left[\left(\int_0^t \Delta\,\dd W\right)\!\left(\int_0^t \Gamma\,\dd W\right)\right]
=\EE\!\left[\int_0^t \Delta(s)\Gamma(s)\,\dd s\right].
\]
\item \textbf{Variazione quadratica:} $\displaystyle \big[\textstyle\int_0^\cdot \Delta\,\dd W\big]_t=\int_0^t \Delta^2(s)\,\dd s.$
\item \textbf{Continuit\`a in $L^2$:} se $\Delta_n\to\Delta$ in $L^2([0,t]\times\Omega)$, allora $\int_0^t \Delta_n\,\dd W \to \int_0^t \Delta\,\dd W$ in $L^2(\Omega)$.
\end{enumerate}

\subsubsection*{Interpretazioni/Commenti}
\begin{itemize}
\item (iii) \emph{Zero mean} $\Rightarrow$ il guadagno stocastico “puro” ha valore atteso nullo: gioco equo dato il passato.
\item (ii)+(v) \emph{Isometria come isometria di Hilbert}: l’It\^o integrale realizza un’isometria $L^2_{\text{prog}} \to \mathcal{M}^2$ (martingale square–integrable); utile per proiezioni/ortogonalit\`a.
\item (vi) La \emph{variazione quadratica} determina il termine $\tfrac12 f''$ nella formula di It\^o: base dell’analisi di SDE e delle PDE associate.
\end{itemize}

\subsubsection*{Applicazione significativa (pricing risk–neutral)}
Sia $S$ soluzione di $\,\dd S_t = r S_t\,\dd t + \sigma(t) S_t\,\dd W_t.$ Allora
\[
\dd\big(e^{-rt}S_t\big)= e^{-rt}\sigma(t)S_t\,\dd W_t
\quad\Rightarrow\quad
e^{-rt}S_t = S_0 + \int_0^t e^{-rs}\sigma(s)S_s\,\dd W_s.
\]
Per (iii) e (iv): $\EE[e^{-rt}S_t]=S_0$ e $(e^{-rt}S_t)$ è martingala $\Rightarrow$ \emph{assenza d’arbitraggio} e base della valutazione risk–neutral.









\newpage
\section{Question 15 -- Itô Integral Expectation}
\textbf{If $I(t)$ is an It\^o integral, what is $\EE(I(t))$? Show one significant application of the property.}

\subsection*{Answer}

\subsubsection*{Statement}
If $I(t)=\displaystyle \int_0^t \Delta(s)\,\dd W(s)$ with $\Delta$ adattato e $\EE\!\big[\int_0^t \Delta^2\,\dd s\big]<\infty$, allora
\[
\boxed{\;\EE[I(t)]=0\;}\quad \text{per ogni } t\ge 0.
\]

\subsubsection*{Reason}
Per definizione via approssimazione con processi semplici (somma di incrementi di $W$ a media zero) e passaggio al limite in $L^2$.

\subsubsection*{Applicazione significativa (media delle soluzioni SDE lineari)}
Considera la SDE lineare (tipo Vasicek con coefficiente generico)
\[
\dd X_t = a(t)X_t\,\dd t + b(t)\,\dd t + \sigma(t)\,\dd W_t,\qquad X_0=x_0.
\]
La soluzione var.\ dei parametri \`e
\[
X_t = \Phi(t)\!\left(x_0 + \int_0^t \Phi(s)^{-1} b(s)\,\dd s\right)
+ \underbrace{\int_0^t \Phi(t)\Phi(s)^{-1}\sigma(s)\,\dd W_s}_{\text{It\^o integrale, media 0}},
\]
dove $\Phi'(t)=a(t)\Phi(t)$, $\Phi(0)=1$. Quindi
\[
\EE[X_t]= \Phi(t)\!\left(x_0 + \int_0^t \Phi(s)^{-1} b(s)\,\dd s\right),
\]
ovvero \emph{il termine diffusivo non contribuisce alla media}. Questo \`e cruciale, ad es., per $\EE[S_t]$ in GBM e per $\EE[r_t]$ nei modelli di tasso.




\newpage
\section{Question 16 -- Itô Isometry}
\textbf{State the isometry property of the It\^o integral. Show an example of application. (hint: compute variance of either the Vasicek or the CIR interest rate)}

\subsection*{Answer}

\subsubsection*{It\^o Isometry}
For $\Delta \in L^2_{\text{prog}}([0,t]\times\Omega)$,
\[
\boxed{\;\EE\!\left[\left(\int_0^t \Delta(s)\,\dd W(s)\right)^{\!2}\right]
= \EE\!\left[\int_0^t \Delta^2(s)\,\dd s\right].\;}
\]
Pi\`u in generale, per $\Delta,\Gamma$,
\[
\EE\!\left[\!\left(\int_0^t \Delta\,\dd W\right)\!\!\left(\int_0^t \Gamma\,\dd W\right)\!\right]
= \EE\!\left[\int_0^t \Delta(s)\Gamma(s)\,\dd s\right].
\]

\subsubsection*{Applicazione: varianza nel modello di Vasicek}
Sia il tasso cort o $r_t$ soluzione di
\[
\dd r_t = a\,(b-r_t)\,\dd t + \sigma\,\dd W_t,\qquad r_0 \in\RR,\; a>0,\; \sigma>0.
\]
La soluzione esplicita \`e
\[
r_t = r_0 e^{-at} + b\big(1-e^{-at}\big) + \sigma \int_0^t e^{-a(t-s)}\,\dd W_s.
\]
Allora
\[
\EE[r_t] = r_0 e^{-at} + b\big(1-e^{-at}\big),
\qquad
\Var(r_t) = \EE\!\left[\left(\sigma \int_0^t e^{-a(t-s)}\,\dd W_s\right)^{\!2}\right].
\]
Per isometria di It\^o,
\[
\Var(r_t) = \sigma^2 \int_0^t e^{-2a(t-s)}\,\dd s
= \frac{\sigma^2}{2a}\Big(1-e^{-2at}\Big).
\]
\textit{Commento.} La varianza cresce a $t\!\to\!\infty$ verso $\sigma^2/(2a)$: mean reversion ($a$ grande) riduce la varianza di lungo periodo.

% (Opzionale) Nota su CIR:
% Per CIR: \dd r_t = a(b-r_t)\dd t + \sigma \sqrt{r_t}\dd W_t, la varianza si ottiene via equazioni per i momenti o Feller; non lineare, ma l'isometria resta centrale nel trattare i termini stocastici.

\newpage
\section{Question 18 -- Itô--Doeblin Formula}
\textbf{State the It\^o--Doeblin formula for the Brownian motion, both in integral and differential form. Compare it to the usual differentiation formula. Prove that the It\^o integral does not coincide with the usual Riemann, or Lebesgue, integral by means of suitable examples.}

\subsection*{Answer}

\subsubsection*{It\^o--Doeblin formula (Brownian motion)}
Let $W_t$ be a Brownian motion and $f\in C^2(\RR)$.

\paragraph{Differential form (time-homogeneous)}
\[
\boxed{\; \dd f(W_t) \;=\; f'(W_t)\,\dd W_t \;+\; \tfrac12\,f''(W_t)\,\dd t \;}
\]

\paragraph{Integral form}
\[
\boxed{\; f(W_t) \;=\; f(W_0) \;+\; \int_0^t f'(W_s)\,\dd W_s \;+\; \tfrac12\int_0^t f''(W_s)\,\dd s \;}
\]

\paragraph{Versione dipendente dal tempo $f\in C^{1,2}([0,T]\times\RR)$}
\[
\boxed{\; \dd f(t,W_t)= f_t(t,W_t)\,\dd t + f_x(t,W_t)\,\dd W_t + \tfrac12 f_{xx}(t,W_t)\,\dd t \;}
\]
\[
\boxed{\; f(t,W_t)= f(0,W_0)+ \int_0^t f_t(s,W_s)\,\dd s + \int_0^t f_x(s,W_s)\,\dd W_s + \tfrac12\int_0^t f_{xx}(s,W_s)\,\dd s \;}
\]

\medskip
\noindent\textit{It\^o table:} \; $\dd t\,\dd t=0,\; \dd t\,\dd W_t=0,\; (\dd W_t)^2=\dd t.$

\subsubsection*{Confronto con la regola di derivazione classica}
Per una funzione liscia $f$ e una curva deterministica $x(t)$: \; $\dd f(x(t)) = f'(x(t))\,\dd x(t)$.
\medskip

Per $x(t)=W_t$ la regola classica fallisce: compare l’ulteriore termine $\tfrac12 f''\,\dd t$ dovuto alla variazione quadratica $[W]_t=t$. Questo è il segno distintivo del calcolo di It\^o.

\subsubsection*{Perché l’integrale di It\^o non coincide con Riemann/Lebesgue: esempi}

\paragraph{Esempio A (catena su $f(x)=x^2$).}
Applicando It\^o a $f(x)=x^2$:
\[
\dd (W_t^2)= 2W_t\,\dd W_t + \dd t
\quad\Rightarrow\quad
\int_0^t W_s\,\dd W_s = \tfrac12\big(W_t^2 - t\big).
\]
Se $\int_0^t W_s\,\dd W_s$ fosse un integrale di Riemann--Stieltjes classico, per integrazione per parti avremmo
$\int_0^t W_s\,\dd W_s = \tfrac12 W_t^2$ (nessun termine $-\,\tfrac12 t$).
L’identità di It\^o mostra l’\emph{extra} $-\tfrac12 t$: i due integrali non coincidono.

\paragraph{Esempio B (inesistenza del Riemann--Stieltjes con integratore $W$).}
Il moto browniano ha variazione totale infinita su ogni intervallo e regolarità di Hölder $<\tfrac12$; l’integrale di Riemann--Stieltjes $\int_0^t W_s\,\dd W_s$ \emph{non esiste} pathwise (fallisce il criterio di Young). L’integrale di It\^o è definito invece come limite $L^2$ di somme prevedibili (left-point), quindi è ben definito e diverso dall’integrale classico.

\paragraph{Esempio C (It\^o vs Lebesgue nel tempo).}
Confronta
\[
I_t:=\int_0^t W_s\,\dd W_s
\qquad\text{e}\qquad
J_t:=\int_0^t W_s\,\dd s .
\]
Si hanno $\EE[I_t]=0$ e, per isometria di It\^o,
\[
\Var(I_t)= \EE\!\left[\int_0^t W_s^2\,\dd s\right]= \int_0^t \EE[W_s^2]\,\dd s = \int_0^t s\,\dd s = \tfrac{t^2}{2}.
\]
Invece
\[
\EE[J_t]=0,\qquad
\Var(J_t)= \iint_{[0,t]^2}\!\!\Cov(W_s,W_u)\,\dd s\,\dd u
= \iint_{[0,t]^2}\!\!\min(s,u)\,\dd s\,\dd u
= \tfrac{t^3}{3}.
\]
Dunque $I_t$ (It\^o, contro $W$) e $J_t$ (Lebesgue, contro $t$) sono variabili aleatorie diverse: l’integrale di It\^o non coincide con l’integrale di Lebesgue nel tempo.

\subsubsection*{Conclusione}
La formula di It\^o--Doeblin aggiunge un termine di drift $\tfrac12 f''\,\dd t$ assente nel calcolo classico: ciò riflette $[W]_t=t$ e rende necessario un \emph{calcolo} ad hoc. Gli esempi mostrano che l’integrale di It\^o è concettualmente e tecnicamente distinto dagli integrali di Riemann--Stieltjes e di Lebesgue.









\newpage    
\section{Question 19 -- Itô Processes}
\subsection*{Itô Processes and Itô--Doeblin Formula}

\begin{definition}[Itô Process]
A stochastic process $X(t)$, $t \geq 0$, is called an \emph{Itô process} if it can be written in the differential form
\[
dX(t) = \mu(t)\,dt + \sigma(t)\,dW(t), \quad X(0) = X_0,
\]
where
\begin{itemize}
    \item $W(t)$ is a Brownian motion,
    \item $\mu(t)$ is a stochastic process (called the \emph{drift}),
    \item $\sigma(t)$ is a stochastic process (called the \emph{diffusion}),
    \item both $\mu$ and $\sigma$ are adapted to the natural filtration of $W(t)$ and satisfy suitable integrability conditions.
\end{itemize}
\end{definition}

\begin{example}[Well-known Itô Processes]
    \begin{enumerate}
    \item \textbf{Brownian Motion:} $dX(t) = dW(t)$.
    \item \textbf{Geometric Brownian Motion (GBM):}
    \[
    dS(t) = \alpha S(t)\,dt + \sigma S(t)\,dW(t), \quad S(0) = S_0,
    \]
    fundamental in financial modeling.
    \item \textbf{Linear SDEs:}
    \[
    dX(t) = \big(\alpha(t)X(t) + \beta(t)\big)\,dt + \sigma(t)\,dW(t).
    \]
    \item \textbf{Vasicek Model (interest rate):}
    \[
    dr(t) = a\big(b-r(t)\big)\,dt + \sigma\,dW(t).
    \]
\end{enumerate}
\end{example}

\begin{theorem}[Itô--Doeblin Formula for Itô Processes]
Let $X(t)$ be an Itô process
\[
dX(t) = \mu(t)\,dt + \sigma(t)\,dW(t),
\]
and let $f \in C^{1,2}(\mathbb{R}_+ \times \mathbb{R})$ (continuously differentiable in $t$ and twice in $x$). Then the process $Y(t) = f(t,X(t))$ satisfies
\[
df(t,X(t)) = \frac{\partial f}{\partial t}(t,X(t))\,dt +
\frac{\partial f}{\partial x}(t,X(t))\,dX(t) +
\frac{1}{2}\frac{\partial^2 f}{\partial x^2}(t,X(t))\,\sigma^2(t)\,dt.
\]
In integral form:
\[
f(t,X(t)) = f(0,X(0)) + \int_0^t \frac{\partial f}{\partial s}(s,X(s))\,ds
+ \int_0^t \frac{\partial f}{\partial x}(s,X(s))\,dX(s)
+ \frac{1}{2}\int_0^t \frac{\partial^2 f}{\partial x^2}(s,X(s))\,\sigma^2(s)\,ds.
\]
\end{theorem}

\begin{example}[Application: Logarithm of a Geometric Brownian Motion]
Consider a GBM $S(t)$ satisfying
\[
dS(t) = \alpha S(t)\,dt + \sigma S(t)\,dW(t).
\]
Let $Y(t) = \ln S(t)$. By Itô's formula, with $f(x) = \ln x$, we compute
\[
f'(x) = \frac{1}{x}, \qquad f''(x) = -\frac{1}{x^2}.
\]
Hence
\[
dY(t) = \frac{1}{S(t)}\,dS(t) - \frac{1}{2}\frac{1}{S^2(t)}\sigma^2 S^2(t)\,dt
= \Big(\alpha - \tfrac{1}{2}\sigma^2\Big)dt + \sigma dW(t).
\]
Therefore $\ln S(t)$ is itself an Itô process with drift $\alpha - \tfrac{1}{2}\sigma^2$ and volatility $\sigma$. This is the key step in deriving the explicit solution of the GBM and in the Black--Scholes option pricing model.
\end{example}










\newpage
\section{Question 20 -- Stochastic Differential Equations}
\subsection*{Stochastic Differential Equations (SDEs)}

\begin{definition}[Stochastic Differential Equation]
Let $W(t)$ be a Brownian motion, $\mu(t,x)$ and $\sigma(t,x)$ deterministic functions, and $x_0 \in \mathbb{R}$ an initial condition. 
A \emph{stochastic differential equation (SDE)} is an equation of the form
\[
dX(t) = \mu(t,X(t))\,dt + \sigma(t,X(t))\,dW(t), 
\quad t \geq 0, \qquad X(0) = x_0.
\]
Here:
\begin{itemize}
    \item $\mu(t,x)$ is called the \emph{drift coefficient},
    \item $\sigma(t,x)$ is called the \emph{diffusion coefficient}.
\end{itemize}
The corresponding \emph{integral formulation} is
\[
X(t) = x_0 + \int_0^t \mu(s,X(s))\,ds + \int_0^t \sigma(s,X(s))\,dW(s).
\]
\end{definition}

\begin{remark}
SDEs describe the dynamics of stochastic processes and extend ordinary differential equations by incorporating randomness via the Brownian motion term.
They are the building blocks of continuous-time models in finance.
\end{remark}

\subsection*{SDEs in Financial Applications}
\begin{enumerate}
    \item \textbf{Geometric Brownian Motion (GBM).}  
    The standard model for stock prices:
    \[
    dS(t) = \alpha S(t)\,dt + \sigma S(t)\,dW(t), \qquad S(0) = S_0 > 0,
    \]
    where $\alpha$ is the mean rate of return and $\sigma$ is the volatility.  
    It admits the explicit solution
    \[
    S(t) = S(0)\exp\!\left( \Big(\alpha - \tfrac{1}{2}\sigma^2\Big)t + \sigma W(t)\right).
    \]

    \item \textbf{Vasicek Interest Rate Model.}  
    A mean-reverting model for the short interest rate:
    \[
    dr(t) = a\big(b-r(t)\big)\,dt + \sigma\,dW(t),
    \]
    where $a > 0$ is the speed of reversion, $b$ the long-term mean, and $\sigma$ the volatility.  
    The solution is
    \[
    r(t) = e^{-at}r(0) + b(1-e^{-at}) + \sigma \int_0^t e^{-a(t-s)}\,dW(s).
    \]

    \item \textbf{Cox–Ingersoll–Ross (CIR) Model.}  
    A refinement of Vasicek ensuring positivity of the short rate:
    \[
    dr(t) = a\big(b-r(t)\big)\,dt + \sigma\sqrt{r(t)}\,dW(t).
    \]
    If the Feller condition $2ab \geq \sigma^2$ holds, then $r(t) \geq 0$ almost surely.
\end{enumerate}






\newpage
\section{Question 21 -- Existence and Uniqueness for SDEs}
\subsection*{21.\;Existence and Uniqueness for SDEs; applications to GBM and linear SDEs}

\begin{theorem}[Esistenza e Unicità per SDE]\label{thm:EU-SDE}
Sia $W(t)$ un moto browniano e siano $\mu,\sigma:\,[0,\infty)\times\mathbb{R}\to\mathbb{R}$ funzioni tali che esiste $K>0$ con, per ogni $t\ge 0$ e $x,y\in\mathbb{R}$,
\begin{align*}
&|\mu(t,x)-\mu(t,y)| \le K|x-y|, \qquad |\sigma(t,x)-\sigma(t,y)| \le K|x-y| \quad\text{(Lipschitz globale in $x$)},\\
&|\mu(t,x)|+|\sigma(t,x)| \le K\,(1+|x|) \qquad\qquad\qquad\qquad\quad\;\;\text{(crescita lineare).}
\end{align*}
Allora l'SDE
\[
dX(t)=\mu(t,X(t))\,dt+\sigma(t,X(t))\,dW(t),\qquad X(0)=x_0,
\]
ha un'unica soluzione forte $X$ adattata a $\sigma(W(s):0\le s\le t)$, con traiettorie continue. % Cfr. Lecture Notes 06, slide "Existence and Uniqueness" (Bjork 5.1).
\end{theorem}

\paragraph{(a) Applicazione al Geometric Brownian Motion (GBM).}
Il GBM è definito da
\[
dS(t)=\alpha\,S(t)\,dt+\sigma\,S(t)\,dW(t),\qquad S(0)=S_0>0,
\]
ossia $\mu(t,x)=\alpha x$, $\sigma(t,x)=\sigma x$ (con $\alpha,\sigma\in\mathbb{R}$ costanti).
Verifica delle ipotesi del Teorema~\ref{thm:EU-SDE}:
\begin{itemize}
\item \emph{Lipschitz in $x$:}\; $|\mu(t,x)-\mu(t,y)|=|\alpha||x-y|\le K|x-y|$ con $K\ge|\alpha|$;\; analogamente $|\sigma(t,x)-\sigma(t,y)|=|\sigma||x-y|\le K|x-y|$ con $K\ge|\sigma|$.
\item \emph{Crescita lineare:}\; $|\mu(t,x)|+|\sigma(t,x)|=|\alpha||x|+|\sigma||x|\le (|\alpha|+|\sigma|)\,(1+|x|)$ scegliendo $K\ge|\alpha|+|\sigma|$.
\end{itemize}
Dunque esiste ed è unica la soluzione forte con traiettorie continue. % Vedi anche la formula esplicita nelle note: $S(t)=S(0)\exp\{(\alpha-\tfrac12\sigma^2)t+\sigma W(t)\}$.

\paragraph{(b) Applicazione alle SDE lineari.}
Consideriamo l'SDE lineare
\[
dX(t)=\big(\alpha(t)X(t)+\beta(t)\big)\,dt+\sigma(t)\,dW(t),\qquad X(0)=x_0,
\]
dove $\alpha,\beta,\sigma:[0,\infty)\to\mathbb{R}$ sono (ad esempio) continue e limitate.
Scrivendo $\mu(t,x)=\alpha(t)x+\beta(t)$ e $\sigma(t,x)=\sigma(t)$, verifichiamo:
\begin{itemize}
\item \emph{Lipschitz in $x$:}\; $|\mu(t,x)-\mu(t,y)|=|\alpha(t)||x-y|\le \|\alpha\|_\infty\,|x-y|$;\; $|\sigma(t,x)-\sigma(t,y)|=0$.
\item \emph{Crescita lineare:}\; $|\mu(t,x)|+|\sigma(t,x)|
\le |\alpha(t)||x|+|\beta(t)|+|\sigma(t)|
\le \big(\|\alpha\|_\infty+\|\beta\|_\infty+\|\sigma\|_\infty\big)\,(1+|x|)$.
\end{itemize}
Quindi le ipotesi del Teorema~\ref{thm:EU-SDE} sono soddisfatte e l'SDE lineare ammette un'unica soluzione forte continua. % Nelle note è anche riportata la formula di variazione delle costanti per la soluzione.









\newpage
\section{Question 22 -- Geometric Brownian Motion}
\subsection*{Geometric Brownian Motion: SDE, soluzione esplicita, valore atteso e varianza}

\paragraph{Definizione (SDE del GBM).}
Un \emph{Geometric Brownian Motion} (GBM) $S(t)$ è soluzione della SDE
\[
dS(t)=\alpha\,S(t)\,dt+\sigma\,S(t)\,dW(t),\qquad S(0)=S_0>0,
\]
dove $\alpha\in\mathbb{R}$ è il tasso medio di crescita, $\sigma>0$ la volatilità, e $W(t)$ un moto browniano.

\paragraph{Proposizione (Soluzione esplicita).}
La soluzione è
\[
S(t)=S_0\,\exp\!\Bigl[\Bigl(\alpha-\tfrac{1}{2}\sigma^2\Bigr)t+\sigma W(t)\Bigr],\qquad t\ge 0.
\]

\begin{proof}[Dimostrazione (completa)]
Poniamo $Y(t)=\ln S(t)$ e applichiamo Itô a $f(x)=\ln x$:
\[
f'(x)=\frac{1}{x},\qquad f''(x)=-\frac{1}{x^2}.
\]
Dalla SDE $dS=\alpha S\,dt+\sigma S\,dW$ otteniamo, con Itô,
\[
dY(t)=\frac{1}{S(t)}\,dS(t)+\frac{1}{2}f''(S(t))\,(dS(t))^2
       =\frac{1}{S}(\alpha S\,dt+\sigma S\,dW)-\frac{1}{2}\frac{1}{S^2}\,\sigma^2 S^2\,dt.
\]
Semplificando e usando $(dW)^2=dt$,
\[
dY(t)=\Bigl(\alpha-\tfrac{1}{2}\sigma^2\Bigr)dt+\sigma\,dW(t).
\]
Integrando da $0$ a $t$,
\[
\ln S(t)-\ln S_0=\Bigl(\alpha-\tfrac{1}{2}\sigma^2\Bigr)t+\sigma W(t),
\]
ed esponenziando si ottiene la tesi.
\end{proof}

\paragraph{Valore atteso (non so se sia il metodo giusto sorry).}
Dalla formula esplicita,
\[
\mathbb{E}[S(t)]=S_0\,e^{(\alpha-\frac{1}{2}\sigma^2)t}\,\mathbb{E}\big[e^{\sigma W(t)}\big].
\]
Poich\'e $W(t)\sim \mathcal{N}(0,t)$, scriviamo l’atteso come integrale gaussiano e \emph{completiamo il quadrato}:
\[
\mathbb{E}\big[e^{\sigma W(t)}\big]
=\int_{-\infty}^{+\infty} e^{\sigma z}\,\frac{1}{\sqrt{2\pi t}}\,e^{-\frac{z^2}{2t}}\,dz
=\frac{1}{\sqrt{2\pi t}}\int_{-\infty}^{+\infty}\exp\!\left[-\frac{1}{2t}\Bigl(z^2-2\sigma tz\Bigr)\right]dz.
\]
Osserviamo che
\[
z^2-2\sigma tz=(z-\sigma t)^2-\sigma^2 t^2,
\]
quindi
\[
\exp\!\left[-\frac{1}{2t}\Bigl(z^2-2\sigma tz\Bigr)\right]
=\exp\!\left[-\frac{(z-\sigma t)^2}{2t}\right]\exp\!\left(\frac{\sigma^2 t}{2}\right).
\]
L’integrale diventa
\[
\mathbb{E}\big[e^{\sigma W(t)}\big]
=\exp\!\Bigl(\tfrac{1}{2}\sigma^2 t\Bigr)\,\frac{1}{\sqrt{2\pi t}}
\int_{-\infty}^{+\infty}\exp\!\left[-\frac{(z-\sigma t)^2}{2t}\right]dz
=\exp\!\Bigl(\tfrac{1}{2}\sigma^2 t\Bigr),
\]
perch\'e l’integrale rimanente è $1$ (densità normale centrata in $\sigma t$ con varianza $t$).
Pertanto
\[
\boxed{\;\mathbb{E}[S(t)]=S_0\,e^{\alpha t}\;}
\]

\paragraph{Varianza.}
Per la varianza calcoliamo prima il secondo momento. Dalla soluzione esplicita,
\[
S(t)^2=S_0^2\,\exp\!\Bigl(2(\alpha-\tfrac{1}{2}\sigma^2)t+2\sigma W(t)\Bigr)
=S_0^2\,e^{2(\alpha-\frac{1}{2}\sigma^2)t}\,e^{2\sigma W(t)}.
\]
Usando di nuovo il calcolo precedente con $2\sigma$ al posto di $\sigma$,
\[
\mathbb{E}\big[e^{2\sigma W(t)}\big]=\exp\!\Bigl(\tfrac{1}{2}(2\sigma)^2 t\Bigr)=e^{2\sigma^2 t},
\]
da cui
\[
\mathbb{E}\big[S(t)^2\big]=S_0^2\,e^{2(\alpha-\frac{1}{2}\sigma^2)t}\,e^{2\sigma^2 t}
=S_0^2\,e^{2\alpha t+\sigma^2 t}.
\]
Infine
\[
\operatorname{Var}\big(S(t)\big)
=\mathbb{E}\big[S(t)^2\big]-\big(\mathbb{E}[S(t)]\big)^2
=S_0^2\,e^{2\alpha t+\sigma^2 t}-S_0^2\,e^{2\alpha t}
=\boxed{\;S_0^2\,e^{2\alpha t}\big(e^{\sigma^2 t}-1\big)\;}.
\]

\paragraph{Osservazioni.}
(1) $S(t)>0$ a.s. e $\ln S(t)\sim \mathcal{N}\!\bigl(\ln S_0+(\alpha-\tfrac{1}{2}\sigma^2)t,\;\sigma^2 t\bigr)$ (legge lognormale).
(2) La formula del valore atteso discende essenzialmente dall’MGF della normale e dalla tecnica del completamento del quadrato, come mostrato sopra.














\newpage
\section{Question 23. + explanation of the Expectation Value of an It\^o Process}

\textbf{Given the It\^o process}
\[
dY(t) = \left(\alpha - \tfrac{1}{2}\sigma^2\right)dt + \sigma\,dW(t), 
\qquad Y(0)=0,
\]
\textbf{define $X(t)=e^{Y(t)}$. Is $X(t)$ an It\^o process? Compute its expected value.}

\subsection*{Answer}

\subsubsection*{Step 1: Identification}
By It\^o’s formula applied to $f(y)=e^y$, one finds
\[
dX(t) = \alpha X(t)\,dt + \sigma X(t)\,dW(t), \qquad X(0)=1.
\]
Therefore $X(t)$ is a \textbf{Geometric Brownian Motion (GBM)} with drift $\alpha$ and volatility $\sigma$.

\subsubsection*{Step 2: Expectation via SDE (exam method)}
We take expectation on both sides:
\[
\mathbb{E}[dX(t)] 
= \mathbb{E}[\alpha X(t)\,dt] + \mathbb{E}[\sigma X(t)\,dW(t)].
\]
Since $\mathbb{E}[X(t)\,dW(t)] = 0$ (property of It\^o integral), we obtain
\[
\mathbb{E}[dX(t)] = \alpha \,\mathbb{E}[X(t)]\,dt.
\]
Defining $m(t) := \mathbb{E}[X(t)]$, this is the ODE
\[
m'(t) = \alpha m(t), \qquad m(0) = 1.
\]
The solution is
\[
m(t) = e^{\alpha t}.
\]

\subsubsection*{Final Result}
\[
\boxed{\;\mathbb{E}[X(t)] = e^{\alpha t}\;}
\]

\subsection*{Additional Example}
Consider the additive SDE
\[
dZ(t) = \mu\,dt + \sigma\,dW(t), \qquad Z(0)=z_0.
\]
Taking expectations:
\[
\mathbb{E}[dZ(t)] = \mu\,dt + \mathbb{E}[\sigma dW(t)] = \mu\,dt.
\]
Hence, with $m(t):=\mathbb{E}[Z(t)]$,
\[
m'(t) = \mu, \qquad m(0)=z_0,
\]
so that
\[
\boxed{\;\mathbb{E}[Z(t)] = z_0 + \mu t.\;}
\]

\subsection*{Comparison}
\begin{itemize}
    \item GBM ($dX=\alpha X dt + \sigma X dW$): exponential growth in expectation, $E[X(t)] = e^{\alpha t}$.
    \item Brownian motion with drift ($dZ=\mu dt + \sigma dW$): linear growth in expectation, $E[Z(t)] = z_0 + \mu t$.
\end{itemize}













\newpage
\section{Question 24 -- Applications of Linear SDEs}
\textbf{Describe in full detail the following application of linear SDEs:
(i) Vasicek model for interest rate; (ii) CIR model for interest rate.
In both cases explain the meaning of the parameters and compute mean, variance, and their long–run limits.}

\subsection*{Answer}

\subsection*{(i) Vasicek model}
\paragraph{Model.}
\[
dr(t)=a\bigl(b-r(t)\bigr)\,dt+\sigma\,dW(t),\qquad r(0)=r_0,\quad a>0,\ \sigma>0,\ b\in\mathbb{R}.
\]

\paragraph{Economic meaning of parameters.}
\begin{itemize}
  \item $a$ (\emph{speed of mean reversion}): how fast $r(t)$ is pulled back to $b$; larger $a$ $\Rightarrow$ faster reversion.
  \item $b$ (\emph{long–run mean/level}): equilibrium level toward which $r(t)$ reverts.
  \item $\sigma$ (\emph{volatility}): instantaneous magnitude of random shocks.
\end{itemize}

\paragraph{Explicit solution (linear SDE integrating factor).}
Let $\phi(t)=e^{at}$. Then
\[
\frac{d}{dt}\bigl(e^{at}r(t)\bigr)=ab\,e^{at}+\sigma e^{at}\dot W(t)
\quad\Longrightarrow\quad
r(t)=r_0e^{-at}+b(1-e^{-at})+\sigma\!\int_0^t e^{-a(t-s)}\,dW(s).
\]

\paragraph{Mean and variance.}
Using $\mathbb{E}\!\left[\int_0^t\! \cdots dW\right]=0$ and Itô isometry,
\[
\boxed{\ \mathbb{E}[r(t)]=r_0e^{-at}+b(1-e^{-at})\ }
\]
\[
\boxed{\ \operatorname{Var}(r(t))=\sigma^2\int_0^t e^{-2a(t-s)}\,ds
=\frac{\sigma^2}{2a}\bigl(1-e^{-2at}\bigr)\ }.
\]

\paragraph{Long–run limits ($t\to\infty$).}
\[
\boxed{\ \lim_{t\to\infty}\mathbb{E}[r(t)]=b,\qquad
\lim_{t\to\infty}\operatorname{Var}(r(t))=\frac{\sigma^2}{2a}\ }.
\]
Hence $r(t)$ is stationary Gaussian in the limit (Ornstein–Uhlenbeck), but it may become negative (drawback for pre-crisis rates modeling).

\medskip

\subsection*{(ii) CIR (Cox–Ingersoll–Ross) model}
\paragraph{Model.}
\[
dr(t)=a\bigl(b-r(t)\bigr)\,dt+\sigma\sqrt{r(t)}\,dW(t),\qquad r(0)=r_0,\quad a>0,\ b>0,\ \sigma>0.
\]

\paragraph{Economic meaning of parameters.}
\begin{itemize}
  \item $a$ (\emph{speed of mean reversion}) and $b$ (\emph{long–run mean}) as in Vasicek.
  \item $\sigma$ (\emph{volatility scale}) now acts through $\sqrt{r(t)}$: volatility is state-dependent, smaller near zero.
\end{itemize}

\paragraph{Positivity (Feller condition).}
If
\[
\boxed{\ 2ab\ \ge\ \sigma^2\ },
\]
then $r(t)$ stays strictly positive (no boundary hitting at $0$). Even when $2ab<\sigma^2$, the square-root term greatly reduces the probability of negative rates versus Vasicek.

\paragraph{Solution representation.}
Although not affine in $r$ (due to $\sqrt{r}$), one can write the variation-of-constants form:
\[
r(t)=r_0 e^{-at}+b(1-e^{-at})+\sigma\int_{0}^{t} e^{-a(t-s)}\sqrt{r(s)}\,dW(s).
\]

\paragraph{Mean and variance.}
A standard computation (taking expectations; Itô isometry with state-dependent diffusion) yields:
\[
\boxed{\ \mathbb{E}[r(t)]=r_0e^{-at}+b(1-e^{-at})\ }
\]
\[
\boxed{\ \operatorname{Var}(r(t))=
\frac{\sigma^2}{a}\,r_0\bigl(e^{-at}-e^{-2at}\bigr)
+\frac{b\,\sigma^2}{2a}\bigl(1-e^{-at}\bigr)^2\ }.
\]

\paragraph{Long–run limits ($t\to\infty$).}
\[
\boxed{\ \lim_{t\to\infty}\mathbb{E}[r(t)]=b,\qquad
\lim_{t\to\infty}\operatorname{Var}(r(t))=\frac{b\,\sigma^2}{2a}\ }.
\]
In the stationary regime (under $2ab>\sigma^2$), $r(t)$ has a Gamma distribution; at finite $t$, $r(t)$ is non-central chi-square—both are positive, matching interest-rate non-negativity.

\medskip

\subsection*{Comparison summary}
\begin{center}
\begin{tabular}{lccc}

Model & SDE & $\ \mathbb{E}[r(t)]\ $ & $\ \operatorname{Var}(r(t))\ $ (long–run)\\

Vasicek & $dr=a(b-r)\,dt+\sigma\,dW$ 
& $r_0e^{-at}+b(1-e^{-at})$ 
& $\dfrac{\sigma^2}{2a}$ \\
CIR & $dr=a(b-r)\,dt+\sigma\sqrt{r}\,dW$ 
& $r_0e^{-at}+b(1-e^{-at})$ 
& $\dfrac{b\,\sigma^2}{2a}$ \\

\end{tabular}
\end{center}

\paragraph{Key takeaways.}
Both models are mean-reverting with the \emph{same} mean function; the variance dynamics differ. Vasicek is Gaussian (may go negative); CIR enforces state-dependent volatility and (under Feller) strict positivity, a desirable feature for short rates.






\newpage
\section{Question: Lebesgue Integral *non credo richiesta}
\textbf{Explain what the Lebesgue integral is, and describe all the main concepts connected with it: 
definition, construction, properties, relation to probability and expected value, 
comparison with the Riemann integral.}

\subsection*{Answer}

\subsubsection*{Motivation}
The Riemann integral is based on partitioning the \emph{domain} of a function and summing rectangles 
under the curve. While powerful, this approach is not suitable for many functions that appear 
in probability and stochastic calculus, especially when random variables are involved.  
The \textbf{Lebesgue integral} generalizes integration by partitioning the \emph{range} of a function 
instead, allowing for much greater flexibility.

\subsubsection*{Measure space setup}
A \textbf{measure space} is a triple $(\Omega,\mathcal{F},\mu)$, where
\begin{itemize}
    \item $\Omega$ is the sample space;
    \item $\mathcal{F}$ is a $\sigma$-algebra of subsets of $\Omega$;
    \item $\mu:\mathcal{F}\to [0,+\infty]$ is a measure, i.e. a countably additive set function.
\end{itemize}
In probability, the measure $\mu$ is the probability measure $P$.

\subsubsection*{Construction of the Lebesgue integral}
The construction proceeds step by step:
\begin{enumerate}[label=\roman*)]
    \item For an \textbf{indicator function} $\chi_A(\omega)$ we define
    \[
    \int_\Omega \chi_A \, d\mu := \mu(A).
    \]
    \item For a \textbf{simple function} $X(\omega)=\sum_{j=1}^n b_j \chi_{A_j}(\omega)$, with $b_j \in \mathbb{R}$ and disjoint $A_j\in\mathcal{F}$,
    \[
    \int_\Omega X\,d\mu := \sum_{j=1}^n b_j \,\mu(A_j).
    \]
    \item For a \textbf{positive measurable function} $X:\Omega\to [0,+\infty]$, we approximate it from below with simple functions $X_n$ and set
    \[
    \int_\Omega X\,d\mu := \sup_n \int_\Omega X_n\,d\mu.
    \]
    \item For a \textbf{general real function} $X$, define the positive and negative parts
    \[
    X^+ = \max\{X,0\}, \qquad X^- = \max\{-X,0\}, \qquad X=X^+-X^-,
    \]
    and declare $X$ integrable if $\int X^+ d\mu < \infty$ and $\int X^- d\mu < \infty$, setting
    \[
    \int_\Omega X\,d\mu := \int_\Omega X^+\,d\mu - \int_\Omega X^-\,d\mu.
    \]
\end{enumerate}

\subsubsection*{Properties of the Lebesgue integral}
\begin{itemize}
    \item \textbf{Linearity:} $\int (aX+bY)\,d\mu = a\int X\,d\mu + b\int Y\,d\mu$.
    \item \textbf{Monotonicity:} If $X\leq Y$ almost surely, then $\int X\,d\mu \leq \int Y\,d\mu$.
    \item \textbf{Fatou’s Lemma:} $\int \liminf X_n \,d\mu \leq \liminf \int X_n \,d\mu$.
    \item \textbf{Monotone Convergence Theorem (MCT):} If $X_n\uparrow X$, then $\int X_n\,d\mu \to \int X\,d\mu$.
    \item \textbf{Dominated Convergence Theorem (DCT):} If $X_n\to X$ and $|X_n|\leq Y$ with $Y$ integrable, then $\int X_n\,d\mu \to \int X\,d\mu$.
\end{itemize}

\subsubsection*{Connection to probability theory}
On a probability space $(\Omega,\mathcal{F},P)$, the expected value of a random variable $X$ is precisely
the Lebesgue integral:
\[
\mathbb{E}[X] = \int_\Omega X(\omega)\,dP(\omega).
\]

\subsubsection*{Comparison with the Riemann integral}
\begin{itemize}
    \item If $f:[a,b]\to\mathbb{R}$ is Riemann-integrable, then it is also Lebesgue-integrable and the two integrals coincide.
    \item The Lebesgue integral extends integration to a much larger class of functions (e.g. discontinuous or non–Riemann integrable functions).
    \item Example: the indicator $\chi_\mathbb{Q}$ of rationals in $[0,1]$ is \emph{not} Riemann integrable but it is Lebesgue integrable with value $0$.
\end{itemize}

\subsubsection*{Why it matters}
The Lebesgue integral is the foundation of modern probability and stochastic calculus.  
It allows us to rigorously define expectations, variances, conditional expectations, and the Itô integral.

\subsection*{Final Summary}

The Lebesgue integral generalizes integration by measuring sets where the function takes values.
It coincides with the expectation in probability theory, and is strictly more powerful than the Riemann integral.

\subsection*{Comparison: Riemann vs Lebesgue Integral}

\begin{center}
\renewcommand{\arraystretch}{1.5}
\begin{tabular}{|p{4cm}|p{5cm}|p{5cm}|}
\hline
\textbf{Aspect} & \textbf{Riemann Integral} & \textbf{Lebesgue Integral} \\
\hline
\textbf{Idea of construction} & 
Partition of the \emph{domain} $[a,b]$ into subintervals; approximate area with rectangles of width $\Delta x$ and height $f(\xi_i)$. &
Partition of the \emph{range} of the function; measure the size of preimages $f^{-1}([y_j,y_{j+1}))$ and weight by the value. \\
\hline
\textbf{Definition for simple functions} &
Not defined separately (works directly with sums over intervals). &
Starts from indicator functions $\chi_A$, extends to simple functions $X=\sum b_j \chi_{A_j}$, then to measurable functions by limits. \\
\hline
\textbf{Integrable functions} &
Continuous (or piecewise continuous) functions on compact intervals; bounded functions with a small set of discontinuities. &
All measurable functions $f$ such that $\int |f|\, d\mu < \infty$. Much larger class (e.g. $\chi_\mathbb{Q}$ is Lebesgue integrable but not Riemann). \\
\hline
\textbf{Convergence theorems} &
No general convergence results (limit of integrals $\neq$ integral of limit in general). &
Powerful tools: Monotone Convergence Theorem (MCT), Fatou’s Lemma, Dominated Convergence Theorem (DCT). \\
\hline
\textbf{Relation to probability} &
Cannot be directly used for expectations of general r.v.s (too restrictive). &
\(\mathbb{E}[X]=\int_\Omega X\,dP\): expectation is a Lebesgue integral. Basis of modern probability and stochastic calculus. \\
\hline
\textbf{Coincidence} &
If $f$ is Riemann integrable, the Riemann and Lebesgue integrals coincide. &
Extends Riemann integral; strictly more general. \\
\hline
\end{tabular}
\end{center}







\newpage
\section{Il modello di Black--Scholes--Merton e l'equazione di Black--Scholes}
\label{sec:bsm-bs}

\subsection{Setup di mercato e ipotesi}
Lavoriamo in un mercato frictionless con scambio continuo, possibilità di short/borrowing illimitati al tasso privo di rischio $r$ costante.
\begin{itemize}
  \item \textbf{Bond (money market):} \ $dB(t)=r\,B(t)\,dt$, $B(0)=1$.
  \item \textbf{Azione (GBM):} \ $dS(t)=\alpha\,S(t)\,dt+\sigma\,S(t)\,dW(t)$, con $\alpha\in\mathbb{R}$, $\sigma>0$ costanti e $W$ moto browniano.
\end{itemize}
Consideriamo una call europea di strike $K$ e scadenza $T$. Il suo prezzo a tempo $t$ quando $S(t)=x$ è una funzione deterministica $c(t,x)$ con condizione terminale
\[
c(T,x) = [x-K]^+.
\]

\subsection*{Passo 1: It\^o su $c(t,S(t))$}
Applicando la formula di It\^o alla composizione $t\mapsto c\big(t,S(t)\big)$:
\[
dc(t,S(t))
= \Big( c_t + \alpha S\,c_x + \tfrac12 \sigma^2 S^2 c_{xx} \Big) dt
\;+\; \sigma S\,c_x\, dW(t),
\]
dove $c_t=\partial c/\partial t$, $c_x=\partial c/\partial x$, $c_{xx}=\partial^2 c/\partial x^2$.

\subsection*{Passo 2: Portafoglio auto-finanziante che replica l'opzione}
Sia $X(t)$ un portafoglio che detiene $\Delta(t)$ azioni e investe il resto nel bond. Essendo auto-finanziante,
\[
dX(t)=\Delta(t)\,dS(t) + r\big(X(t)-\Delta(t)S(t)\big)\,dt,
\]
ossia
\[
dX(t) = \big[rX(t)+\Delta(t)(\alpha-r)S(t)\big]dt \;+\; \Delta(t)\sigma S(t)\,dW(t).
\]
Richiediamo \emph{replica esatta}: $X(t)=c\big(t,S(t)\big)$ per ogni $t\in[0,T]$.

\subsection*{Passo 3: Sconto al tasso privo di rischio}
Passiamo alle quantità scontate (per eliminare il drift risk-free):
\[
\begin{aligned}
d\!\big(e^{-rt}X(t)\big) &= e^{-rt}\big[(\alpha-r)\Delta S\,dt + \sigma \Delta S\,dW\big],\\
d\!\big(e^{-rt}c(t,S(t))\big) &= e^{-rt}\Big[\big(c_t+\alpha S c_x+\tfrac12\sigma^2 S^2 c_{xx}-r c\big)dt
+ \sigma S c_x\,dW\Big].
\end{aligned}
\]
Poich\'e $e^{-rt}X(t)\equiv e^{-rt}c(t,S(t))$, i coefficienti di $dW$ e $dt$ devono coincidere identicamente.

\subsection*{Passo 4: Matching dei coefficienti e PDE di Black--Scholes}
Dal matching del termine casuale ($dW$) si ottiene la \emph{delta-hedge rule}
\[
\Delta(t)=c_x\big(t,S(t)\big).
\]
Sostituendo nel matching dei termini deterministici ($dt$):
\[
c_t + \alpha S c_x + \tfrac12 \sigma^2 S^2 c_{xx} - r c \;=\; (\alpha - r) S c_x,
\]
da cui si semplifica il drift $\alpha S c_x$ e si ottiene la \textbf{PDE di Black--Scholes}
\begin{equation}
\label{eq:BS-PDE}
c_t(t,x) \;+\; r\,x\,c_x(t,x) \;+\; \tfrac12\,\sigma^2 x^2\,c_{xx}(t,x) \;-\; r\,c(t,x) \;=\; 0,
\qquad 0\le t<T,\; x>0.
\end{equation}

\subsection{Condizioni al contorno: enunciato e interpretazione}
Per rendere ben posto il problema su $\{(t,x):\,0\le t\le T,\,x\ge 0\}$ imponiamo:
\begin{enumerate}
  \item \textbf{Condizione terminale (payoff):} \ $c(T,x)=[x-K]^+$. \\
  \emph{Interpretazione:} a scadenza il derivato vale esattamente il payoff contrattuale.
  \item \textbf{Valore a sottostante nullo:} \ $c(t,0)=0$ per $0\le t\le T$. \\
  \emph{Interpretazione:} se il titolo vale zero, una call è priva di valore (coerente anche con \eqref{eq:BS-PDE} in $x=0$).
  \item \textbf{Condizione asintotica (crescita lineare corretta):}
  \[
  \lim_{x\to\infty}\Big(c(t,x)-\big(x-e^{-r(T-t)}K\big)\Big)=0 \quad \text{per ogni } 0\le t<T.
  \]
  \emph{Interpretazione:} per $x\gg K$, l'esercizio è (quasi) certo: la call si comporta come ``azione meno valore attuale dello strike''.
\end{enumerate}

\subsection*{Osservazioni}
\begin{itemize}
  \item La scelta $\Delta=c_x$ elimina il rischio diffusivo (termini in $dW$) del replicante; il portafoglio risultante è privo di rischio e deve rendere $r$, da cui \eqref{eq:BS-PDE}.
  \item In misura risk--neutral il drift dell'azione diventa $r$; per Feynman--Kac
  \[
  c(t,x)=e^{-r(T-t)}\,\mathbb{E}^{\!*}\!\big[(S(T)-K)^+\mid S(t)=x\big],
  \]
  che soddisfa la stessa PDE con le condizioni sopra, portando alla formula chiusa classica.
\end{itemize}

\subsection*{Formula sheet (richiamo rapido)}
\noindent\textbf{PDE di Black--Scholes:}\quad
$c_t + r x c_x + \tfrac12 \sigma^2 x^2 c_{xx} - r c = 0$,
con
$c(T,x)=[x-K]^+$,
$c(t,0)=0$,
$\;\lim_{x\to\infty}\!\big(c-(x-e^{-r(T-t)}K)\big)=0$.

\medskip
\noindent\textbf{Delta-hedge:}\quad $\Delta(t)=c_x\big(t,S(t)\big)$.

\subsection{Le principali Greeks}

Una volta nota la formula di Black--Scholes per la call europea, si possono calcolare le derivate parziali del prezzo rispetto alle variabili principali. Queste sensibilità sono dette \emph{Greeks}. Le più importanti sono:

\begin{itemize}
  \item \textbf{Delta ($\Delta$):}
  \[
  \Delta(t,x) = \frac{\partial c}{\partial x}(t,x) = N(d_+).
  \]
  Misura la sensibilità del prezzo dell’opzione rispetto al prezzo del sottostante. È sempre positiva, quindi il prezzo della call cresce al crescere di $x$.

  \item \textbf{Gamma ($\Gamma$):}
  \[
  \Gamma(t,x) = \frac{\partial^2 c}{\partial x^2}(t,x)
  = \frac{1}{x\sigma\sqrt{T-t}}\,N'(d_+).
  \]
  Indica come varia la Delta al variare del sottostante. È sempre positiva, quindi la funzione prezzo è convessa in $x$.

  \item \textbf{Theta ($\Theta$):}
  \[
  \Theta(t,x) = \frac{\partial c}{\partial t}(t,x)
  = -rK e^{-r(T-t)}N(d_-) - \frac{\sigma x}{2\sqrt{T-t}}\,N'(d_+).
  \]
  Misura la sensibilità del prezzo rispetto al passare del tempo. È generalmente negativa, descrivendo il decadimento temporale del valore della call.
\end{itemize}


\section{Setting e significato del modello di Black--Scholes--Merton}

\subsection{Il modello di mercato}
Consideriamo un mercato frictionless, senza costi di transazione e con possibilità di trading continuo.
Due sono gli asset fondamentali:
\begin{itemize}
  \item \textbf{Bond (money market account)}: 
  \[
  dB(t) = r B(t)\,dt, \quad B(0)=1,
  \]
  dove $r \geq 0$ è il tasso di interesse privo di rischio.
  \item \textbf{Azione (stock)}: segue un moto browniano geometrico (GBM)
  \[
  dS(t) = \alpha S(t)\,dt + \sigma S(t)\,dW(t),
  \]
  con $\alpha \in \mathbb R$ (drift), $\sigma > 0$ (volatilità) e $W(t)$ moto browniano.
\end{itemize}

\subsection{Il contratto: Call Europea}
Una call europea con strike $K$ e scadenza $T$ ha payoff
\[
c(T,S(T)) = [S(T) - K]^+.
\]
Definiamo $c(t,x)$ il prezzo al tempo $t$ quando $S(t)=x$.

\subsection{Derivazione dell'equazione di Black--Scholes}
Applicando It\^o alla funzione $c(t,S(t))$:
\[
dc = \big(c_t + \alpha x c_x + \tfrac12 \sigma^2 x^2 c_{xx}\big)\,dt + \sigma x c_x\, dW.
\]

Consideriamo ora un portafoglio replicante $X(t)$ con $\Delta(t)$ azioni e resto in bond:
\[
dX = \Delta\,dS + r(X-\Delta S)\,dt.
\]

La condizione di \emph{replica} impone $X(t)=c(t,S(t))$ per ogni $t$.  
Confrontando le dinamiche (dopo sconto al tasso $r$) si ottiene:
\[
\Delta(t) = c_x(t,S(t)), \quad\quad
c_t + r x c_x + \tfrac12 \sigma^2 x^2 c_{xx} - r c = 0.
\]

\begin{equation}
\label{eq:BS}
\boxed{ \; c_t(t,x) + r x c_x(t,x) + \tfrac12 \sigma^2 x^2 c_{xx}(t,x) - r c(t,x) = 0, \quad 0 \leq t < T, \; x > 0. \;}
\end{equation}

\subsection{Condizioni al contorno}
Affinché il problema sia ben posto imponiamo:
\begin{enumerate}
  \item \textbf{Condizione terminale (payoff)}:
  \[
  c(T,x) = [x-K]^+.
  \]
  \emph{Interpretazione:} a scadenza l'opzione vale esattamente il payoff contrattuale.
  
  \item \textbf{Valore a sottostante nullo}:
  \[
  c(t,0)=0 \quad \forall t \in [0,T].
  \]
  \emph{Interpretazione:} se l'azione vale zero, la call non ha valore.
  
  \item \textbf{Condizione asintotica (transversalità)}:
  \[
  \lim_{x\to\infty} \Big( c(t,x) - \big(x - e^{-r(T-t)}K\big)\Big)=0.
  \]
  \emph{Interpretazione:} per $x \gg K$ la call equivale a possedere l'azione meno il valore attuale dello strike.
\end{enumerate}

\subsection{Significato economico}
\begin{itemize}
  \item La \emph{delta hedge rule} $\Delta=c_x$ elimina il rischio stocastico: il portafoglio replicante diventa privo di rischio.
  \item L'equazione \eqref{eq:BS} rappresenta la condizione di assenza di arbitraggio: un portafoglio privo di rischio deve crescere al tasso $r$.
  \item La valutazione non dipende da $\alpha$, ma solo da $r,\sigma,K,T$: è il principio di \emph{risk-neutral pricing}.
  \item Con le condizioni al contorno, la PDE ha un'unica soluzione: la formula di Black--Scholes per la call.
\end{itemize}




\end{document}